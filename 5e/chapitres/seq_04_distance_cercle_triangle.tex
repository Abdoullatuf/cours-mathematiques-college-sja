% Séquence 4 : Distance, cercle et triangles
\setseqtitle{Distance, cercle et triangles}
\chapter{Distance, cercle et triangles}

\begin{objectifsbox}
\textbf{Objectifs d'apprentissage.} À l'issue de la séquence, l'élève sera capable de :
\begin{itemize}
\item Calculer des distances
\item Construire des cercles
\item Reconnaître différents types de triangles
\end{itemize}
\end{objectifsbox}

\section{La distance}

\begin{definitionbox}
La \textbf{distance} entre deux points est la longueur du segment qui les relie.
\end{definitionbox}

\begin{examplebox}
\textbf{Exemple :} La distance entre les points A et B se note AB.
\end{examplebox}

\section{Le cercle}

\begin{definitionbox}
Un \textbf{cercle} est l'ensemble des points situés à la même distance d'un point appelé centre.
\end{definitionbox}

\begin{proprietebox}
\textbf{Propriété :} Tous les points du cercle sont à égale distance du centre.
\end{proprietebox}

\begin{exercisebox}
\textbf{Exercice d'application :}
Tracer un cercle de centre O et de rayon 3 cm.
\end{exercisebox}
