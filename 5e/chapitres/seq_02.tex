% Séquence 2 : Points et droites
\setseqtitle{Points et droites}
\chapter{Points et droites}

\begin{objectifsbox}
\textbf{Objectifs d'apprentissage.} À l'issue de la séquence, l'élève sera capable de :
\begin{itemize}
\item Représenter des points et des droites
\item Utiliser le vocabulaire géométrique approprié
\item Construire des figures géométriques simples
\end{itemize}
\end{objectifsbox}

\section{Représentation des points}

\begin{definitionbox}
Un \textbf{point} est représenté par une croix ou un petit cercle plein.
\end{definitionbox}

\begin{examplebox}
\textbf{Exemple :} Le point A se note A et se représente par $\times$ ou $\bullet$
\end{examplebox}

\section{Représentation des droites}

\begin{definitionbox}
Une \textbf{droite} est une ligne droite qui s'étend à l'infini dans les deux sens.
\end{definitionbox}

\begin{proprietebox}
\textbf{Propriété :} Par deux points distincts, il passe une seule droite.
\end{proprietebox}

\begin{exercisebox}
\textbf{Exercice d'application :}
Tracer la droite passant par les points A et B.
\end{exercisebox}
