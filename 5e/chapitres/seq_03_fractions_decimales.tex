% Séquence 3 : Fractions décimales et nombres décimaux
\setseqtitle{Fractions décimales et nombres décimaux}
\chapter{Fractions décimales et nombres décimaux}

\begin{objectifsbox}
\textbf{Objectifs d'apprentissage.} À l'issue de la séquence, l'élève sera capable de :
\begin{itemize}
\item Reconnaître les fractions décimales
\item Convertir entre fractions décimales et nombres décimaux
\item Utiliser les nombres décimaux dans des calculs
\end{itemize}
\end{objectifsbox}

\section{Les fractions décimales}

\begin{definitionbox}
Une \textbf{fraction décimale} est une fraction dont le dénominateur est une puissance de 10.
\end{definitionbox}

\begin{examplebox}
\textbf{Exemples :} $\frac{3}{10}$, $\frac{25}{100}$, $\frac{7}{1000}$
\end{examplebox}

\section{Les nombres décimaux}

\begin{definitionbox}
Un \textbf{nombre décimal} est un nombre qui peut s'écrire avec une virgule.
\end{definitionbox}

\begin{proprietebox}
\textbf{Propriété :} $\frac{3}{10} = 0,3$ et $\frac{25}{100} = 0,25$
\end{proprietebox}

\begin{exercisebox}
\textbf{Exercice d'application :}
Convertir $\frac{7}{100}$ en nombre décimal.
\end{exercisebox}
