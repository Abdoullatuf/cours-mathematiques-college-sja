% Séquence 1 : Les nombres entiers
\setseqtitle{Les nombres entiers}
\chapter{Les nombres entiers}

\begin{objectifsbox}
\textbf{Objectifs d'apprentissage.} À l'issue de la séquence, l'élève sera capable de :
\begin{itemize}
\item Reconnaître et utiliser les nombres entiers
\item Comparer et ordonner les nombres entiers
\item Effectuer des calculs avec les nombres entiers
\end{itemize}
\end{objectifsbox}

\section{Qu'est-ce qu'un nombre entier ?}

\begin{definitionbox}
Un \textbf{nombre entier} est un nombre qui peut s'écrire sans virgule ni fraction.
\end{definitionbox}

\begin{examplebox}
\textbf{Exemples de nombres entiers :}
\begin{itemize}
\item 0, 1, 2, 3, 4, 5, ...
\item -1, -2, -3, -4, -5, ...
\end{itemize}
\end{examplebox}

\section{Comparaison des nombres entiers}

\begin{proprietebox}
\textbf{Propriété :} Sur une droite graduée, plus un nombre est à droite, plus il est grand.
\end{proprietebox}

\begin{exercisebox}
\textbf{Exercice d'application :}
Comparer les nombres suivants : 7, -3, 0, -8, 12
\end{exercisebox}
