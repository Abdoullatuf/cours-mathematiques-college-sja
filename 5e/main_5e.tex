\documentclass[12pt,a4paper]{book}
\newcommand{\niveau}{5\textsuperscript{e}}
% ================================
% PRÉAMBULE COMMUN — Mathématiques (6e, 5e, 4e)
% ================================

% Encodage et langue
\usepackage[T1]{fontenc}
\usepackage[french]{babel}

% Typographie moderne
\usepackage{newtxtext,newtxmath}

% Configuration des pages
\usepackage{geometry}
\geometry{top=2.5cm, bottom=2.5cm, left=2.5cm, right=2.5cm}

% Amélioration du rendu
\usepackage{microtype}

% Personnalisation des titres
\usepackage{titlesec}
\titleformat{\chapter}[hang]{\huge\bfseries}{\thechapter.}{0.6em}{}
\titleformat{\section}[hang]{\Large\bfseries}{\thesection}{0.6em}{}
\titleformat{\subsection}[hang]{\large\bfseries}{\thesubsection}{0.6em}{}

% Ajustement de l'espacement des titres
\titlespacing{\chapter}{0pt}{-10pt}{20pt}
\titlespacing{\section}{0pt}{12pt}{8pt}
\titlespacing{\subsection}{0pt}{4pt}{2pt}

% En-têtes et pieds de page
\usepackage{fancyhdr}

% Packages mathématiques
\usepackage{amsmath, mathtools}
\usepackage{siunitx}

% Graphiques
\usepackage{tikz}
\usetikzlibrary{angles, quotes, calc, arrows.meta, shapes.geometric}
\usepackage{pgfplots}
\pgfplotsset{compat=1.18}

% Liens hypertexte
\usepackage{hyperref}
\hypersetup{
    colorlinks=true,
    linkcolor=blue!60!black,
    urlcolor=blue!60!black,
    citecolor=blue!60!black
}

% Boîtes colorées avec tcolorbox
\usepackage[most]{tcolorbox}

% Configuration globale des boîtes
\tcbset{
    rounded corners,
    boxsep=2ex,
    top=1.5ex,
    bottom=1.5ex,
    left=2.5ex,
    right=7ex,
    before skip=0pt,
    after skip=1.5ex,
    width=\textwidth,
    boxrule=1pt
}



% Définition des environnements personnalisés
\newtcolorbox{definitionbox}{
    colback=orange!5!white,
    colframe=orange!70!black,
    title={\textbf{Définition}},
    fonttitle=\bfseries,
    coltitle=black
}

\newtcolorbox{examplebox}{
    colback=green!5!white,
    colframe=green!60!black,
    title={\textbf{Exemple}},
    fonttitle=\bfseries,
    coltitle=black
}

\newtcolorbox{exercisebox}{
    colback=purple!5!white,
    colframe=purple!70!black,
    title={\textbf{Exercices}},
    fonttitle=\bfseries,
    coltitle=black
}

\newtcolorbox{objectifsbox}{
    colback=teal!5!white,
    colframe=teal!70!black,
    title={\textbf{Objectifs}},
    fonttitle=\bfseries,
    coltitle=black
}

\newtcolorbox{proprietebox}{
    colback=red!5!white,
    colframe=red!70!black,
    title={\textbf{Propriété}},
    fonttitle=\bfseries,
    coltitle=black
}

\newtcolorbox{activitybox}{
    colback=blue!5!white,
    colframe=blue!70!black,
    title={\textbf{Activité}},
    fonttitle=\bfseries,
    coltitle=black
}

\newtcolorbox{remarkbox}{
    colback=yellow!5!white,
    colframe=yellow!70!black,
    title={\textbf{Remarque}},
    fonttitle=\bfseries,
    coltitle=black
}

\newtcolorbox{quizbox}{
    colback=cyan!5!white,
    colframe=cyan!70!black,
    title={\textbf{Quiz}},
    fonttitle=\bfseries,
    coltitle=black
}

\newtcolorbox{methodebox}{
    colback=purple!5!white,
    colframe=purple!70!black,
    title={\textbf{Méthode}},
    fonttitle=\bfseries,
    coltitle=black
}

% Commande personnalisée pour les trous
\newcommand{\trous}[1]{\makebox[#1]{\rule{0pt}{1.2ex}\dotfill}}

% Variable pour stocker le titre de la séquence
\newcommand{\seqtitle}{}
\newcommand{\setseqtitle}[1]{\renewcommand{\seqtitle}{#1}}

% Mise en page de l'en-tête et du pied
\pagestyle{fancy}
\setlength{\headheight}{16pt} % FIX: évite le warning fancyhdr
\fancyhf{}
\lhead{Mathématiques \niveau{} -- 2025--2026}
\rhead{Seq.~\thechapter~--~\seqtitle}
\cfoot{\thepage}

% Listes compactes
\usepackage{enumitem}
\setlist[itemize]{left=1.2em}
\setlist[enumerate]{left=1.5em}

% Configuration des labels personnalisés pour enumitem
\SetEnumitemKey{a}{label=\alph*)}
\SetEnumitemKey{1}{label=\arabic*)}

% Définition des styles de listes personnalisés
\newlist{exerciselist}{enumerate}{1}
\setlist[exerciselist]{label=\alph*)}

\newlist{quizlist}{enumerate}{1}
\setlist[quizlist]{label=\arabic*)}

% Packages pour tableaux et présentations
\usepackage{longtable}
\usepackage{array}
\usepackage{booktabs}


\title{Cours de Mathématiques — Classe de 5\textsuperscript{e}\\[0.4em]\large Année scolaire 2025–2026}
\author{Abdoullatuf Maoulida}
\date{\today}

\begin{document}
\maketitle
\tableofcontents
\cleardoublepage

%\part{Nombres et calculs}
% Séquence 1 : Les nombres entiers
\setseqtitle{Les nombres entiers}
\chapter{Les nombres entiers}

\section{Rang des chiffres}

\begin{definitionbox}
	\textbf{Chiffres et valeur}
	\begin{itemize}[label = \textbullet]
		\item 0, 1, 2, 3, 4, 5, 6, 7, 8, 9 sont les dix \textbf{chiffres} qui permettent d'écrire tous les nombres.
		\item Chaque chiffre a une \textbf{valeur} en fonction de sa position dans le nombre.
	\end{itemize}
\end{definitionbox}

On peut utiliser un tableau de numération pour visualiser le rang d'un chiffre.

\begin{center}
	\newcolumntype{C}{>{\centering\arraybackslash}p{0.8cm}}
	\begin{tabular}{|C|C|C|C|C|C|C|C|C|C|C|C|}
		\hline
		\multicolumn{3}{|c|}{\textbf{Milliards}} & 
		\multicolumn{3}{|c|}{\textbf{Millions}} & 
		\multicolumn{3}{|c|}{\textbf{Milliers}} & 
		\multicolumn{3}{|c|}{\textbf{Unités}} \\
		\hline
		\textbf{c} & \textbf{d} & \textbf{u} & 
		\textbf{c} & \textbf{d} & \textbf{u} & 
		\textbf{c} & \textbf{d} & \textbf{u} & 
		\textbf{c} & \textbf{d} & \textbf{u} \\
		\hline
		& & & & & & & & & & & \\
		\hline
	\end{tabular}
\end{center}

\textbf{Remarque :} Lorsqu'on écrit un nombre en chiffres, il faut laisser un espace entre les classes. Par exemple le nombre suivant 25204879603 s'écrit: \trous{2.5cm}.

\section{Décomposition décimale}

On peut donner la décomposition décimale de 3 584 :

\begin{examplebox}
	3 584 = (\trous{1cm} $\times$ \trous{2cm}) + (\trous{1cm} $\times$ \trous{1.5cm}) + (\trous{0.5cm} $\times$ \trous{1cm}) + \trous{0.5cm} $\times$ \trous{0.5cm}
\end{examplebox}



\textbf{Attention !} Pour le nombre 3 584, le \textbf{chiffre} des centaines est \trous{2cm} mais le \textbf{nombre} de centaines est \trous{3cm} (il y a \trous{2cm} centaines dans le nombre 3584).

En effet :
\trous{14cm}

\trous{15.5cm}


\begin{examplebox}
Dans le nombre 25 803,\\ 

le chiffre des dizaines est \trous{1.5cm}; le nombre de dizaines est \trous{2.5cm}\\

le chiffre des centaines est \trous{1.5cm} ; le nombre de centaines est \trous{2.5cm}
\end{examplebox}

\section{Écriture en toutes lettres}

\begin{examplebox}
	\begin{itemize}[label = \textbullet]
		\item 1823 : Mille-huit-cent-vingt-trois (pas de << s >> à << cent >>, ni à << vingt >> car ils sont suivis d'autres chiffres !)
		\item 2087 : Deux-mille-quatre-vingt-sept (le mot << mille >> est invariable, et toujours pas de << s >> à << vingt >>...)
		\item 600 : Six-cents (ici on met bien un << s >> car il n'y a plus rien derrière !)
		\item 680 : Six-cent-quatre-vingts (pas de << s >> à << cent >>, mais un << s >> obligatoire à << vingt >> car le nombre se termine par 80).
	\end{itemize}
\end{examplebox}

Voici les règles correspondantes à ces exemples :

\begin{itemize}
	\item Le mot << mille >> est invariable ; les mots << million >> et << milliard >>, cependant, s'accordent et prennent donc un << \textbf{s} >> au pluriel.
	\item Les mots << cent >> et << vingt >> prennent un << \textbf{s} >> au pluriel uniquement lorsqu'ils sont à la fin du nombre.
	\item \textbf{Exemples :} 300 : \trous{5cm} \qquad 420 : \trous{5cm}
	\item Le mot << vingt >> ne s'utilise au pluriel (avec un << s >>) que si un nombre se finit par 80 (quatre-vingts).
	\item Les tirets sont mis entre chaque mot d'un nombre qui se présente sous forme composée. Avec des nombres entiers, il y aura donc des tirets partout !
	\item \textbf{Exemples :} \\
	79 : \trous{6cm} \\
	1031 : \trous{6cm}
\end{itemize}

\section{Demi-droite graduée}

\begin{definitionbox}
	\textbf{Demi-droite graduée}
	
	On appelle \textbf{demi-droite graduée} une demi-droite sur laquelle on fixe :
	\begin{itemize}[label = \textbullet]
		\item Un point appelé \textbf{origine de la demi-droite}
		\item \textbf{Un sens} représenté par une flèche
		\item \textbf{Une unité de longueur} que l'on reporte régulièrement à partir de l'origine.
	\end{itemize}
\end{definitionbox}

\begin{figure}[h]
	\centering
	\includegraphics[width=0.6\linewidth]{../../assets/images/6e/seq_01/demi-droite-graduee.png}
	\caption{Demi-droite graduée}
	\label{fig:demi-droite-graduee}
\end{figure}

\begin{proprietebox}
	Sur une demi-droite graduée,
	\begin{itemize}[label = \textbullet]
	 \item Chaque point est repéré par \trous{3cm} appelé \trous{3cm} de ce point.
	 \item A chaque nombre correspond \trous{3cm} unique.
	\end{itemize}
\end{proprietebox}

\begin{examplebox}
	% Schéma de droite graduée avec TikZ
	\begin{center}
		\begin{tikzpicture}[scale=0.8]
			% Droite graduée
			\draw[thick, ->] (0,0) -- (12,0);
			
			% Graduations
			\foreach \x in {0,1,...,11} {
				\draw (\x,-0.1) -- (\x,0.1);
				\node[below] at (\x,-0.3) {\x};
			}
			
			% Point A
			\node[circle,fill=black,inner sep=2pt] at (3.5,0) {};
			\node[above] at (3.5,0.2) {$A$};
			
			% Point B
			\node[circle,fill=black,inner sep=2pt] at (7,0) {};
			\node[above] at (7.2,0.2) {$B$};
		\end{tikzpicture}
	\end{center}
	
	Sur cette demi-droite graduée, le point $A$ a pour abscisse \trous{1.5cm} et le point $B$ a pour abscisse \trous{1.5cm}.	
\end{examplebox}


\begin{center}
	\includegraphics[width=1\linewidth]{../../assets/images/6e/seq_01/attention-demi-dte-graduee}
\end{center}


\section{Exercices d'application}

\begin{exercisebox}
\textbf{Exercice 1 :} Écris en toutes lettres les nombres suivants :
\begin{enumerate}
	\item 1 234
	\item 5 678
	\item 12 345
	\item 100 000
\end{enumerate}

\textbf{Exercice 2 :} Place les points $A$, $B$, $C$ et $D$ d'abscisses respectives 2, 7, 4 et 9 sur une demi-droite graduée.


\end{exercisebox}



 % Les nombres entiers

\cleardoublepage
\appendix
\chapter{Progression annuelle (récapitulatif)}
Cette progression correspond à la répartition établie pour l'année 2025–2026.

\begin{center}
\begin{tabular}{|l|l|}
\hline
\textbf{Période} & \textbf{Séquences}\\ \hline
Période 1 (6 semaines) & S01 -- Les nombres entiers, S02 -- Points et droites, S03 -- Fractions décimales et nombres décimaux\\ \hline
Période 2 (7 semaines) & S04 -- Distance, cercle et triangles, S05 -- Notion de proportionnalité, S06 -- Notion de probabilités, S07 -- Angles et rapporteur\\ \hline
Période 3 (6 semaines) & S08 -- Opérations avec les nombres décimaux, S09 -- La médiatrice d'un segment, S10 -- La division, S11 -- Symétrie axiale\\ \hline
Période 4 (7 semaines) & S12 -- Fraction partage et comparaison de fractions, S13 -- Unités de longueur, de masse et de contenance, S14 -- Calculer avec les angles, S15 -- Nombres en écriture fractionnaire\\ \hline
Période 5 (6 semaines) & S16 -- Proportionnalité et pourcentages, S17 -- Déterminer des probabilités et des issues, S18 -- Aires et périmètres, S19 -- Heures et durées, S20 -- Solides et volumes\\ \hline
\end{tabular}
\end{center}

\end{document}
