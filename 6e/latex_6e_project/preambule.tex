% ================================
% PRÉAMBULE — Mathématiques 6e
% ================================
%\usepackage[utf8]{inputenc} % <- si vous compilez avec XeLaTeX/LuaLaTeX, commentez cette ligne
\usepackage[T1]{fontenc}
\usepackage[french]{babel}
\usepackage{lmodern}
\usepackage{geometry}
\geometry{top=2.5cm,bottom=2.5cm,left=2.3cm,right=2.3cm}
\usepackage{microtype}
\usepackage{titlesec}
\usepackage{fancyhdr}
\usepackage{enumitem}
\usepackage{amsmath,amssymb,amsfonts,mathtools}
\usepackage{siunitx}
\usepackage{graphicx}
\usepackage{tikz}
\usetikzlibrary{angles,quotes,calc,arrows.meta,shapes.geometric}
\usepackage{pgfplots}
\pgfplotsset{compat=1.18}
\usepackage{hyperref}
\usepackage{xcolor}
\hypersetup{
	colorlinks=true,
	linkcolor=blue!50!black,
	urlcolor=blue!50!black,
	citecolor=blue!50!black
}

% Environnements jolis pour Définition / Exemple / Exercice
\usepackage[most]{tcolorbox}
\tcbset{
    rounded corners,
    boxsep=2ex,
    top=1.5ex,
    bottom=1.5ex,
    left=2.5ex,
    right=7ex,
    before skip=1ex,
    after skip=2ex,
    width=\textwidth,
    boxrule=1pt
}
\newtcolorbox{definitionbox}{
    colback=orange!5!white,
    colframe=orange!70!black,
    title={\textbf{Définition}},
    fonttitle=\bfseries,
    coltitle=black
}
\newtcolorbox{examplebox}{
    colback=green!5!white,
    colframe=green!60!black,
    title={\textbf{Exemple}},
    fonttitle=\bfseries,
    coltitle=black
}
\newtcolorbox{exercisebox}{
    colback=purple!5!white,
    colframe=purple!70!black,
    title={\textbf{Exercice}},
    fonttitle=\bfseries,
    coltitle=black
}
\newtcolorbox{objectifsbox}{
    colback=teal!5!white,
    colframe=teal!70!black,
    title={\textbf{Objectifs}},
    fonttitle=\bfseries,
    coltitle=black
}
\newtcolorbox{proprietebox}{
    colback=red!5!white,
    colframe=red!70!black,
    title={\textbf{Propriété}},
    fonttitle=\bfseries,
    coltitle=black
}

% Commande personnalisée pour les trous
\newcommand{\trous}[1]{\dotfill\hspace{#1}\dotfill}

% Variable pour stocker le titre de la séquence
\newcommand{\seqtitle}{}
\newcommand{\setseqtitle}[1]{\renewcommand{\seqtitle}{#1}}

% Mise en page de l'en-tête et du pied
\pagestyle{fancy}
\setlength{\headheight}{16pt} % FIX: évite le warning fancyhdr
\fancyhf{}
\lhead{Mathématiques 6\textsuperscript{e} -- 2025--2026}
\rhead{Seq.~\thechapter~--~\seqtitle}
\cfoot{\thepage}

% Titres moins encombrants
\titleformat{\chapter}[hang]{\huge\bfseries}{\thechapter.}{0.6em}{}
\titleformat{\section}[hang]{\Large\bfseries}{\thesection}{0.6em}{}
\titleformat{\subsection}[hang]{\large\bfseries}{\thesubsection}{0.6em}{}

% Listes compactes
\setlist[itemize]{left=1.2em}
\setlist[enumerate]{left=1.5em}

% ================================
% Commandes pour opérations posées
% ================================
\makeatletter

% largeur fixe pour les chiffres
\newcommand{\digit}[1]{\makebox[1.5ex][c]{#1}}

% Environnement générique pour opérations posées
\newenvironment{operation}[1][5]{%
	\arraycolsep=0.2em
	\renewcommand{\arraystretch}{1.2}%
	\begin{array}{*{#1}{c}}}{\end{array}}

% Addition posée (résultat vide avec \trous)
\newcommand{\addition}[2]{%
	\begin{operation}[10]
		& #1 \\
		+  & #2 \\
		\hline
		& \multicolumn{10}{c}{\trous{2cm}} \\
	\end{operation}
}

% Soustraction posée (résultat vide avec \trous)
\newcommand{\soustraction}[2]{%
	\begin{operation}[10]
		& #1 \\
		-  & #2 \\
		\hline
		& \multicolumn{10}{c}{\trous{2cm}} \\
	\end{operation}
}

\makeatother
