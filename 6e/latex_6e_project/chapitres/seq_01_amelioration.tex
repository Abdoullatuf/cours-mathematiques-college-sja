% Séquence 1 : Les nombres entiers
\setseqtitle{Les nombres entiers}
\chapter{Les nombres entiers}

\chapterobjectives{
  \begin{itemize}
    \item Comprendre la valeur des chiffres selon leur position.
    \item Décomposer un nombre en base 10.
    \item Écrire des nombres en toutes lettres.
    \item Utiliser une demi-droite graduée pour représenter des nombres.
    \item Comparer des nombres entiers.
  \end{itemize}
}

\section{Rang des chiffres}

\begin{definitionbox}
  \textbf{Chiffres et valeur}
  \begin{itemize}[label = \textbullet]
    \item Les chiffres 0, 1, 2, 3, 4, 5, 6, 7, 8, 9 permettent d'écrire tous les nombres.
    \item La valeur d'un chiffre dépend de sa position dans le nombre.
  \end{itemize}
\end{definitionbox}

\marginnote{\small Conseil : Pense à vérifier la position de chaque chiffre !}[1cm]

\begin{center}
  \newcolumntype{C}{>{\centering\arraybackslash}p{1.2cm}}
  \begin{tabular}{|C|C|C|C|C|C|C|C|C|C|C|C|}
    \hline
    \multicolumn{3}{|c|}{\textbf{Classe des milliards}} & 
    \multicolumn{3}{|c|}{\textbf{Classe des millions}} & 
    \multicolumn{3}{|c|}{\textbf{Classe des milliers}} & 
    \multicolumn{3}{|c|}{\textbf{Classe des unités}} \\
    \hline
    \textbf{c} & \textbf{d} & \textbf{u} & 
    \textbf{c} & \textbf{d} & \textbf{u} & 
    \textbf{c} & \textbf{d} & \textbf{u} & 
    \textbf{c} & \textbf{d} & \textbf{u} \\
    \hline
    & & & & & & & & & & & \\
    \hline
  \end{tabular}
\end{center}

\textbf{Remarque :} Les nombres s'écrivent avec un espace entre les classes. Par exemple, 25204879603 s'écrit : \trous{2.5cm}.

\begin{jeubox}
  \textbf{Jeu : Complète le tableau de numération} \\
  Place les chiffres du nombre 45 678 912 dans le tableau ci-dessus.
\end{jeubox}

\section{Décomposition décimale}

\begin{examplebox}
  Pour le nombre 3 584 : \\
  3 584 = (3 $\times$ 1000) + (5 $\times$ 100) + (8 $\times$ 10) + (4 $\times$ 1)
\end{examplebox}

\marginnote{\small Astuce : Écris d'abord les milliers, puis les centaines, etc.}[0.5cm]

\textbf{Attention !} Dans 3 584, le chiffre des centaines est 5, mais le nombre de centaines est 35 (il y a 35 centaines dans 3 584). \\
Explication : 3 584 = 3 $\times$ 1000 + 5 $\times$ 100 = 30 centaines + 5 centaines = 35 centaines.

\section{Écriture en toutes lettres}

\begin{examplebox}
  \begin{itemize}[label = \textbullet]
    \item 1823 : Mille-huit-cent-vingt-trois
    \item 2087 : Deux-mille-quatre-vingt-sept
    \item 600 : Six-cents
    \item 680 : Six-cent-quatre-vingts
  \end{itemize}
\end{examplebox}

\begin{proprietebox}
  \begin{itemize}
    \item \textbf{Mille} est invariable ; \textbf{million} et \textbf{milliard} prennent un \textbf{s} au pluriel.
    \item \textbf{Cent} et \textbf{vingt} prennent un \textbf{s} uniquement à la fin du nombre.
    \item \textbf{Vingt} prend un \textbf{s} pour les nombres se terminant par 80.
    \item Les tirets sont utilisés entre chaque mot d'un nombre composé.
    \item Exemples : 300 : trois-cents \quad 420 : quatre-cent-vingt \quad 79 : soixante-dix-neuf \quad 1031 : mille-trente-et-un
  \end{itemize}
\end{proprietebox}

\section{Demi-droite graduée}

\begin{definitionbox}
  \textbf{Demi-droite graduée} \\
  Une demi-droite graduée est une demi-droite avec :
  \begin{itemize}[label = \textbullet]
    \item Une \textbf{origine} (point de départ).
    \item Un \textbf{sens} (indiqué par une flèche).
    \item Une \textbf{unité de longueur} reportée régulièrement.
  \end{itemize}
\end{definitionbox}

\begin{center}
  \begin{tikzpicture}[scale=1]
    \draw[thick, -Stealth] (0,0) -- (12,0);
    \foreach \x in {0,1,...,11} {
      \draw (\x,-0.15) -- (\x,0.15);
      \node[below, font=\small] at (\x,-0.3) {\x};
    }
    \node[circle, fill=black, inner sep=2pt, label=above:$A$] at (3.5,0) {};
    \node[circle, fill=black, inner sep=2pt, label=above:$B$] at (7.2,0) {};
  \end{tikzpicture}
\end{center}

\marginnote{\small Conseil : Vérifie l’unité avant de placer un point !}[0.5cm]

\textbf{Exemple :} Sur cette demi-droite, l'abscisse de $A$ est 3,5 et celle de $B$ est 7,2.

\section{Comparaison de nombres entiers}

\begin{definitionbox}
  \textbf{Comparaison de nombres entiers} \\
  Pour comparer deux nombres entiers :
  \begin{itemize}
    \item Les placer sur une demi-droite graduée.
    \item Comparer le nombre de chiffres.
    \item Comparer les chiffres de gauche à droite.
  \end{itemize}
\end{definitionbox}

\begin{examplebox}
  Comparons 2 847 et 2 853 :
  \begin{itemize}
    \item Les deux nombres ont 4 chiffres.
    \item Milliers : 2 = 2.
    \item Centaines : 8 = 8.
    \item Dizaines : 4 < 5.
    \item Donc 2 847 < 2 853.
  \end{itemize}
\end{examplebox}

\chaptersummary{
  \begin{itemize}
    \item Les chiffres ont une valeur selon leur position (unités, dizaines, centaines, etc.).
    \item La décomposition décimale exprime un nombre comme une somme de puissances de 10.
    \item Les nombres en lettres suivent des règles précises pour \textbf{mille}, \textbf{cent}, et \textbf{vingt}.
    \item Une demi-droite graduée aide à visualiser les nombres.
    \item La comparaison des nombres peut se faire par position ou chiffres.
  \end{itemize}
}

\section{Exercices d'application}

\begin{exercisebox}
  \textbf{Exercice 1 :} Écris en toutes lettres les nombres suivants :
  \begin{enumerate}
    \item 1 234
    \item 5 678
    \item 12 345
    \item 100 000
  \end{enumerate}
\end{exercisebox}

\begin{exercisebox}
  \textbf{Exercice 2 :} Place les points $A$, $B$, $C$ et $D$ d'abscisses respectives 2, 7, 4 et 9 sur une demi-droite graduée.
\end{exercisebox}

\begin{exercisebox}
  \textbf{Exercice 3 :} Compare les nombres suivants en utilisant les symboles <, > ou = :
  \begin{enumerate}
    \item 1 234 \trous{1cm} 1 243
    \item 5 678 \trous{1cm} 5 678
    \item 12 345 \trous{1cm} 12 354
  \end{enumerate}
\end{exercisebox}

\begin{jeubox}
  \textbf{Jeu : Chasse aux nombres} \\
  Trouve un nombre entier qui satisfait les conditions suivantes :
  \begin{itemize}
    \item Il a 4 chiffres.
    \item Le chiffre des centaines est 7.
    \item Il est plus grand que 5 000 mais plus petit que 6 000.
  \end{itemize}
  Écris ce nombre en chiffres et en lettres.
\end{jeubox}

\section{Évaluation rapide (5 à 10 min)}

\begin{exercisebox}
  \textbf{Mini-quiz :}
  \begin{enumerate}
    \item Écris en toutes lettres : 8 765
    \item Quel est le chiffre des centaines dans 12 345 ?
    \item Place le point $E$ d'abscisse 6 sur une demi-droite graduée
  \end{enumerate}
\end{exercisebox}