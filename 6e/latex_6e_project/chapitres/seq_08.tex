\setseqtitle{Opérations avec les nombres décimaux}
\chapter{Opérations avec les nombres décimaux}
\label{chap:seq08}

\begin{objectifsbox}
\textbf{Objectifs d'apprentissage.} À l'issue de la séquence, l'élève sera capable de :
\begin{itemize}
  \item Additionner et soustraire des nombres décimaux
  \item Multiplier des nombres décimaux
  \item Poser correctement les opérations avec des nombres décimaux
  \item Calculer des ordres de grandeur
\end{itemize}
\end{objectifsbox}

\section*{Pré-requis}
\begin{itemize}
  \item Connaissance des nombres entiers et de leurs opérations
  \item Maîtrise de la numération décimale
  \item Compréhension de la valeur des chiffres selon leur position
\end{itemize}

\section{Addition et soustraction avec des nombres décimaux}

\begin{definitionbox}
	\textbf{Addition et soustraction}
	
	On calcule une \textbf{\trous{3cm}} lorsqu'on ajoute deux nombres, et une \textbf{\trous{3cm}} lorsqu'on en soustrait deux.
	
	Le résultat d'une addition est une \textbf{\trous{2.5cm}}, celui d'une soustraction une \textbf{\trous{2.5cm}}.
	
	Les nombres calculés ensemble s'appellent les \textbf{\trous{2.5cm}}.
\end{definitionbox}

\begin{examplebox}
	\begin{center}
		\begin{tabular}{p{0.45\textwidth}p{0.45\textwidth}}
			\textbf{21,5 + 12,3 = 33,8} (\trous{2.5cm}) & \textbf{21,5 - 12,3 = 9,2} (\trous{2.5cm}) \\
			On dit que << 33,8 est la somme de 21,5 et 12,3 >> & On dit que << 9,2 est la différence de 21,5 par 12,3 >> \\
		\end{tabular}
	\end{center}
	
	Dans les deux cas, les deux nombres \textbf{21,5} et \textbf{12,3} sont les \textbf{termes} du calcul.
\end{examplebox}

\begin{proprietebox}
On peut échanger les termes d'une addition sans modifier son résultat. On dit que l'addition est \textbf{\trous{3cm}}.
\end{proprietebox}

\textbf{ATTENTION :} Ce n'est pas vrai pour une soustraction !

% Exemple 1 : opération en ligne dans une boxe
\begin{examplebox}
	\begin{center}
		\begin{tabular}{p{0.45\textwidth} p{0.45\textwidth}}
			\textbf{Exemple 1 (opération en ligne)} : \\
			$8,5 + 7,2 + 2,1 + 3,4$ & $8,5 - 3,2$ \\
			$= \underline{\trous{1cm} + \trous{1cm}} + \underline{\trous{1cm} + \trous{1cm}}$ & $= 5,3$ \\
			$= 10,6 + 10,6$ & (attention : on ne sait pas encore calculer $3,2 - 8,5$) \\
			$= 21,2$ & \\
		\end{tabular}
	\end{center}
\end{examplebox}

\vspace{0.5cm}

% Exemple 2 : opération posée dans une boxe
\begin{examplebox}
	\begin{center}
		\begin{tabular}{p{0.45\textwidth} p{0.45\textwidth}}
			\textbf{Exemple 2 (opération posée)} : \underline{\textbf{28,4 + 84,39}} 
			& 
			\underline{\textbf{20,18 - 19,45}} \\
			
			\begin{tabular}{r}
				84,39 \\
				+ 28,40 \\
				\hline
				\trous{1.5cm}
			\end{tabular} &
			\begin{tabular}{r}
				20,18 \\
				- 19,45 \\
				\hline
				\trous{1.5cm}
			\end{tabular} \\
			
		\end{tabular}
	\end{center}
\end{examplebox}


\textbf{Remarques :}
\begin{itemize}
	\item Les mots << \textbf{addition} >> et << \textbf{soustraction} >> désignent des opérations, tandis que les mots << \textbf{somme} >> et << \textbf{différence} >> désignent des nombres (des résultats).
	\item Pour poser une addition ou une soustraction de nombres décimaux, il faut impérativement aligner les nombres par la droite et aligner les virgules.
	\item On peut ajouter des zéros à droite d'un nombre décimal sans changer sa valeur (exemple : 28,4 = 28,40).
\end{itemize}

\section{Multiplication avec des nombres décimaux}

\begin{definitionbox}
	\textbf{Multiplication}
	
	Dans une \textbf{multiplication}, on multiplie des \trous{2.5cm}, et le résultat est un \trous{2cm}.
\end{definitionbox}

\begin{examplebox}
	On dit que 60,5 est le \trous{2.5cm} de 12,1 par 5
\end{examplebox}

\begin{proprietebox}
On peut échanger l'ordre des facteurs sans changer le résultat. On dit que la multiplication est \trous{3cm}.
\end{proprietebox}

\textbf{Méthode :} Poser une multiplication avec des nombres décimaux.

\textbf{Exemple :} $25,1 \times 7,53$

\begin{center}
	\begin{tabular}{r}
		25,1 \\
		$\times$ 7,53 \\
		\hline
		\trous{2cm} \\
		\trous{2cm} \\
		\trous{2cm} \\
		\hline
		\trous{2cm}
	\end{tabular}
\end{center}

\textbf{Règle pour placer la virgule :} Le nombre de chiffres après la virgule dans le résultat est égal à la somme du nombre de chiffres après la virgule dans chaque facteur.

\section{Ordre de grandeur}

\begin{definitionbox}
	\textbf{Ordre de grandeur}
	
	Un \textbf{ordre de grandeur} d'un nombre est un nombre proche de celui-ci et facile à utiliser en calcul mental.
	
	\textbf{Remarque :} Un ordre de grandeur n'est pas unique : on peut donner des ordres de grandeur différents selon la précision voulue.
\end{definitionbox}

\begin{examplebox}
	La population française était de 67 063 703 habitants en 2020. Un ordre de grandeur de cette population est \trous{3cm} (on pourrait aussi choisir 100 millions ou 67 millions).
\end{examplebox}

\begin{examplebox}
Pour calculer $24,7 \times 3,8$, on peut d'abord estimer :

• 24,7 $\approx$ 25 (ordre de grandeur)

• 3,8 $\approx$ 4 (ordre de grandeur)

• Donc $24,7 \times 3,8 \approx 25 \times 4 = 100$

Le résultat exact sera proche de 100.
\end{examplebox}

\section{Exercices d'entraînement}

\begin{exercisebox}
\textbf{Exercices d'application :} Pose et calcule les opérations suivantes :
\begin{enumerate}
	\item 45,7 + 23,8
	\item 67,2 - 34,5
	\item $12,3 \times 4,6$
\end{enumerate}
\end{exercisebox}

\begin{exercisebox}
\textbf{Exercices d'application :} Donne un ordre de grandeur de chaque calcul, puis calcule le résultat exact :
\begin{enumerate}
	\item 23,4 + 45,7
	\item 89,2 - 12,8
	\item $15,6 \times 3,2$
\end{enumerate}
\end{exercisebox}

\section{Évaluation rapide (5 à 10 min)}

\begin{exercisebox}
\textbf{Évaluation rapide :}
\begin{enumerate}
	\item Quel est le résultat de 12,5 + 8,7 ?
	\item Calcule 45,2 - 23,8
	\item Donne un ordre de grandeur de $34,7 \times 2,1$
\end{enumerate}
\end{exercisebox}
