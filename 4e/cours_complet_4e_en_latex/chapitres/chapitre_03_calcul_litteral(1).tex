\chapter{Calcul littéral (1)}
\section{Introduction :} Le calcul littéral est l'utilisation de lettres pour représenter des nombres dans des calculs. Cela permet de généraliser des formules et des opérations sans avoir à les refaire à chaque nouvelle valeur, et de modéliser de nombreuses situations : formules géométriques (aires, périmètres, volumes), formules physiques (distance, durée, vitesse), calculs économiques (coûts en fonction de quantités), etc.

\textbf{Problématique du chapitre :} Comment utiliser des lettres dans les calculs pour généraliser des situations et simplifier des expressions mathématiques ?

\section{Écrire et utiliser une formule littérale}
\begin{definition}{Expression littérale} Une \textbf{expression littérale} est une expression mathématique contenant une ou plusieurs lettres (appelées \textbf{variables}), en plus de nombres. Chaque lettre représente un nombre qui peut varier. Par exemple, $5a + 7$ est une expression littérale où $a$ est une variable.
\end{definition}

On rencontre fréquemment des \textbf{formules littérales} dans lesquelles les lettres permettent de calculer différentes valeurs. Par convention, dans une expression littérale, on peut \textbf{omettre le signe « $\times$ »} dans les cas suivants : \begin{itemize}[label=\textbullet] 
\item entre un nombre et une lettre (on écrit $4a$ au lieu de $4 \times a$) ; 
\item entre deux lettres (on écrit $ab$ au lieu de $a \times b$) ; \item entre un nombre et une parenthèse ou entre deux parenthèses (on écrit $3(x+5)$ au lieu de $3 \times (x+5)$). 
\end{itemize}

\begin{exemple} \textbf{Formule du périmètre d'un rectangle :}\ Soit un rectangle de longueur $L$ et de largeur $l$. Le périmètre $P$ de ce rectangle se calcule avec la formule littérale :
P=2\times\left(L+l\right).
Par exemple, si $L = 8$ cm et $l = 5$ cm, alors
P=2\times\left(8+5\right)=2\times13=26\mathrm{\ cm}.
Dans cette formule, $L$ et $l$ sont des variables qui peuvent prendre différentes valeurs selon la taille du rectangle. On peut ainsi utiliser la formule pour n'importe quelles longueurs $L$ et $l$. 
\end{exemple}

Il est important de bien distinguer l'expression littérale (la formule générale) de son utilisation pour des valeurs numériques particulières. Une formule littérale permet de \textbf{modéliser} une situation de calcul de façon générale ; on peut ensuite remplacer les lettres par des nombres pour effectuer un calcul concret.

\section{Réduire et simplifier une expression littérale}
Réduire (ou \textbf{simplifier}) une expression littérale, c'est la réécrire de la manière la plus simple possible, en rassemblant ce qui peut l'être. En particulier, on peut \textbf{regrouper les termes semblables} d'une somme algébrique. Un \textbf{terme} est chaque élément séparé par un + ou un $-$ dans une somme : par exemple dans l'expression $4x + 7 - 2x + 3$, les termes sont $4x$, $7$, $-2x$ et $3$. Les \textbf{termes semblables} sont ceux qui comportent la même partie littérale (mêmes lettres aux mêmes puissances). On peut donc les additionner ou les soustraire entre eux.

\begin{methode}{Réduire une expression littérale}
\begin{itemize} 
\item \textbf{Identifier les termes semblables :} repérer les termes qui ont la même partie littérale (même variable).
\item \textbf{Combiner leurs coefficients :} additionner ou soustraire les nombres qui multiplient ces lettres (appelés \textbf{coefficients}). Les autres termes (avec une partie littérale différente ou constants) restent inchangés. 
\item \textbf{Réécrire l'expression simplifiée :} avec les termes semblables regroupés et combinés. On obtient ainsi une expression plus compacte, avec moins de termes. 
\end{itemize} 
\end{methode}

\begin{exemple} Simplifions l'expression $4x + 7 - 2x + 3$.\ On regroupe les termes en $x$ d'une part, et les termes constants d'autre part : $(4x - 2x) + (7 + 3)$. En effectuant les calculs : $4x - 2x = 2x$ et $7 + 3 = 10$.\ Ainsi, $4x + 7 - 2x + 3$ se réduit à $\boxed{2x + 10}$. 
\end{exemple}

\textbf{Remarques :} 
\begin{itemize}[label=\textbullet] 
\item Si une lettre est écrite sans coefficient, cela signifie que le coefficient est 1. Par exemple $x$ signifie $1x$, et $-x$ signifie $-1x$. 
\item On ne peut \textbf{pas} combiner des termes qui n'ont pas la même partie littérale : par exemple, $5a + 3b$ reste tel quel car $a$ et $b$ sont différents. De même, $2x + 5$ ne peut pas être simplifié davantage car l'un des termes comporte $x$ et l'autre non. 
\end{itemize}


\section{Calculer la valeur d'une expression littérale pour une valeur donnée}
Il est souvent utile de \textbf{calculer la valeur} d'une expression littérale en remplaçant la variable par un nombre particulier. Par exemple, si l'on a une formule générale pour un calcul, on peut vouloir la \og tester \fg{} ou l'utiliser avec des valeurs précises.

\begin{methode}{Calculer la valeur d'une expression littérale} \begin{itemize} 
\item \textbf{Remplacer la ou les lettres par les valeurs données :} on substitue chaque variable par le nombre indiqué, en le plaçant de préférence entre parenthèses, surtout si la valeur est négative ou si des calculs de puissance sont présents. 
\item \textbf{Effectuer les calculs pas à pas :} en respectant l'ordre des opérations (parenthèses, puis puissances, multiplication et division, puis addition et soustraction). Simplifier progressivement pour obtenir le résultat final. \end{itemize} 
\end{methode}

\begin{exemple} Soit l'expression $E = 5x - 3$. Pour $x = 4$, on remplace $x$ par 4 : $E = 5 \times 4 - 3 = 20 - 3 = 17$. \ Si maintenant $F = 3 - x$ et que $x = -4$, on remplace $x$ par $-4$ (en le mettant entre parenthèses) : $F = 3 - (-4) = 3 + 4 = 7$. \end{exemple}
Dans le calcul littéral, remplacer la variable par un nombre s'appelle également \textbf{évaluer l'expression} en ce nombre.


\section{Développer une expression : la simple distributivité}
\begin{propriete}{Distributivité de la multiplication sur l'addition} Pour tous nombres $a$, $b$ et $c$ :
a\times\left(b+c\right)=a\times b+a\times c.
De même, pour la soustraction :
a\times\left(b-c\right)=a\times b-a\times c.
Autrement dit, \textbf{multiplier une somme (ou une différence) par un nombre revient à multiplier chaque terme de cette somme par ce nombre}, puis à additionner (ou soustraire) les produits obtenus. 
\end{propriete}

\begin{definition}{Développer une expression} \textbf{Développer} une expression, c'est utiliser la distributivité pour transformer un produit en une somme. On passe d'une écriture factorisée à une écriture développée (sans parenthèses). Par exemple, développer $3(x + 4)$ donne $3x + 12$. 
\end{definition}

\begin{methode}{Développer une expression littérale} Pour développer une expression de la forme $k \times (A + B)$ (où $k$ peut être un nombre ou une expression simple, et $A$ et $B$ des termes quelconques) : 
\begin{itemize} 
\item \textbf{Identifier le facteur à distribuer :} repérer le nombre (ou le facteur) situé devant la parenthèse. 
\item \textbf{Multiplier chaque terme à l'intérieur de la parenthèse par ce facteur :} attention à bien conserver les signes. 
\item \textbf{Écrire la somme des produits obtenus :} on obtient l'expression développée (sans parenthèses). \ \textit{Si possible, on peut ensuite réduire l'expression obtenue en regroupant les termes semblables.} 
\end{itemize} 
\end{methode}

\begin{exemple} Développons l'expression $7(x - 5)$ :\ On distribue le $7$ sur chaque terme à l'intérieur de la parenthèse : $7 \times x$ et $7 \times (-5)$.\ On obtient $7 \times x + 7 \times (-5) = 7x - 35$. Ainsi, $7(x - 5)$ développé s'écrit $\boxed{7x - 35}$.\[0.2em] \textit{Vérification numérique :} si $x = 2$, $7(x-5) = 7 \times (-3) = -21$. Du côté développé, $7x - 35$ pour $x=2$ donne $7 \times 2 - 35 = 14 - 35 = -21$. On retrouve le même résultat, ce qui confirme la distributivité. 
\end{exemple}
\bigskip

\section{Applications et résolution de problèmes}
\subsection{Problèmes concrets utilisant des formules littérales}
\begin{exercice}{1 - Coût d'une consommation électrique} Un appareil électrique consomme $P$ kilowatts (kW) de puissance et fonctionne pendant $t$ heures. On utilise la formule littérale $E = P \times t$ pour calculer l'énergie consommée $E$ (en kWh).
\medskip
\textbf{Questions :} 
\begin{enumerate}[label=\alph*)] 
\item Quelle est l'énergie consommée par un radiateur de $2{,}5$~kW qui fonctionne pendant $6$~h ? 
\item La climatisation d'une salle a consommé $18$~kWh en fonctionnant pendant $4$~h. Quelle était sa puissance $P$ (en kW) ? 
\end{enumerate}

\textbf{Solution :} 
\begin{itemize} 
\item a) On applique la formule $E = P \times t$ avec $P = 2,5$ et $t = 6$ : $E = 2,5 \times 6 = 15$~kWh. 
\item b) D'après la formule, $P = \dfrac{E}{t}$. On remplace $E = 18$ et $t = 4$ : $P = \dfrac{18}{4} = 4,5$~kW. 
\end{itemize} 
\end{exercice}

\begin{exercice}{2 - Jeu du nombre mystère} On propose le jeu suivant : \og Choisissez un nombre, multipliez-le par $2$, ajoutez $5$, puis enlevez le nombre de départ. \fg{}
\medskip
\textbf{Questions :} 
\begin{enumerate}[label=\alph*)] 
\item Si on appelle $x$ le nombre de départ, écrire une expression littérale $E$ correspondant au résultat obtenu à la fin du jeu (après ces opérations). 
\item Simplifier l'expression $E$. Que remarque-t-on ? 
\item Vérifier avec quelques valeurs de $x$ (par exemple $x=3$, $x=10$, $x=-4$) que le résultat du jeu est toujours le même, quel que soit le nombre de départ. 
\end{enumerate}

\textbf{Solution :} 
\begin{itemize} 
\item a) Le nombre de départ est $x$. Après les opérations : on le multiplie par $2$ $\Rightarrow 2x$, puis on ajoute $5$ $\Rightarrow 2x + 5$, puis on enlève le nombre de départ ($-x$). Donc l'expression finale est $E = 2x + 5 - x$. 
\item b) On réduit l'expression : $2x + 5 - x = (2x - x) + 5 = 1x + 5 = x + 5$. 
\item c) Si $x=3$, $E = 3 + 5 = 8$. Si $x=10$, $E = 10 + 5 = 15$. Si $x=-4$, $E = -4 + 5 = 1$. Dans chaque cas, le résultat du jeu est $x + 5$, c'est-à-dire \textbf{5 de plus que le nombre de départ}. Le résultat dépend du nombre initial (il n'est pas constant), mais la règle générale du jeu est modélisée par l'expression $x+5$. 
\end{itemize} 
\end{exercice}

\subsection{Utilisation de la distributivité}
\begin{exercice}{3 - Calcul mental astucieux} En utilisant la distributivité, on peut faciliter certains calculs de tête. Par exemple : pour calculer $12 \times 49$, on peut voir $49$ comme $50 - 1$.
\medskip
\textbf{Questions :} 
\begin{enumerate}[label=\alph*)] 
\item Calculer $12 \times 49$ en utilisant l'astuce ci-dessus (distributivité). 
\item De même, calculer $7 \times 101$ en voyant $101$ comme $100 + 1$. 
\end{enumerate}

\textbf{Solution :} 
\begin{itemize} 
\item a) $12 \times 49 = 12 \times (50 - 1) = 12 \times 50 - 12 \times 1 = 600 - 12 = 588$. \item b) $7 \times 101 = 7 \times (100 + 1) = 7 \times 100 + 7 \times 1 = 700 + 7 = 707$. \end{itemize} \end{exercice}

\section{Exercices d'entraînement}
\subsection{Exercices de base}
\begin{exercice}{4} Simplifier les expressions littérales suivantes : 
\begin{enumerate}[label=\alph*)] 
\item $5x + 8x$ 
\item $7a - 3a + 4$ 
\item $2a + 3b - a + 5$ 
\item $6x - 3 - 4x + 1$ 
\end{enumerate} 
\end{exercice}

\begin{exercice}{5} Calculer la valeur des expressions suivantes pour les valeurs indiquées : 
\begin{enumerate}[label=\alph*)] 
\item $4x - 7$ pour $x = 5$ 
\item $2a + 5b$ pour $a = 2$ et $b = 3$ 
\item $5(x + 3)$ pour $x = 4$ 
\end{enumerate} 
\end{exercice}

\begin{exercice}{6} Développer (c'est-à-dire supprimer les parenthèses en utilisant la distributivité) les expressions suivantes : 
\begin{enumerate}[label=\alph*)] 
\item $5(x + 3)$ 
\item $-2(x + 4)$ 
\item $4(7 - x)$ 
\end{enumerate} 
\end{exercice}

\subsection{Exercices d'approfondissement}
\begin{exercice}{7} Simplifier les expressions suivantes en effectuant les développements nécessaires : 
\begin{enumerate}[label=\alph*)] 
\item $3x + 2(x + 5)$ 
\item $10 - 2(x - 4)$ 
\item $4(2x - 1) + 3x$ 
\end{enumerate} 
\end{exercice}

\begin{exercice}{8 - Tour de magie} Choisissez un nombre (notez-le $n$). Effectuez successivement les opérations suivantes : \begin{itemize} 
\item Multipliez $n$ par $2$. 
\item Ajoutez $8$ au résultat. 
\item Prenez la moitié de ce nouveau résultat. 
\item Soustrayez le nombre de départ $n$. 
\end{itemize} 
Quel résultat obtient-on en fonction de $n$ ? Simplifier l'expression littérale obtenue pour le montrer, puis vérifier avec quelques valeurs de $n$ que ce résultat est toujours le même. \end{exercice}


\section{Correction des exercices}
\noindent \textbf{Exercice 4 :} 
\begin{enumerate}[label=\alph*)] 
\item $5x + 8x = 13x$. 
\item $7a - 3a + 4 = (7a - 3a) + 4 = 4a + 4$. 
\item $2a + 3b - a + 5 = (2a - a) + 3b + 5 = a + 3b + 5$. (On ne peut pas simplifier davantage car $a$ et $b$ sont différents.) \item $6x - 3 - 4x + 1 = (6x - 4x) + (-3 + 1) = 2x - 2$. \end{enumerate}

\medskip

\noindent \textbf{Exercice 5 :} 
\begin{enumerate}[label=\alph*)] 
\item $4x - 7$ pour $x=5$ : $4 \times 5 - 7 = 20 - 7 = 13$. 
\item $2a + 5b$ pour $a=2$, $b=3$ : $2 \times 2 + 5 \times 3 = 4 + 15 = 19$. 
\item $5(x+3)$ pour $x=4$ : $5 \times (4+3) = 5 \times 7 = 35$. \end{enumerate}

\medskip

\noindent \textbf{Exercice 6 :} 
\begin{enumerate}[label=\alph*)] 
\item $5(x+3) = 5 \times x + 5 \times 3 = 5x + 15$. 
\item $-2(x+4) = -2 \times x + (-2) \times 4 = -2x - 8$. 
\item $4(7 - x) = 4 \times 7 + 4 \times (-x) = 28 - 4x$. \end{enumerate}

\medskip

\noindent \textbf{Exercice 7 :} 
\begin{enumerate}[label=\alph*)] 
\item $3x + 2(x+5) = 3x + 2x + 10 = (3x+2x) + 10 = 5x + 10$. 
\item $10 - 2(x-4) = 10 - 2x + 8 = -2x + 18$ (on peut aussi écrire $18 - 2x$). 
\item $4(2x - 1) + 3x = (8x - 4) + 3x = 8x + 3x - 4 = 11x - 4$. \end{enumerate}

\medskip

\noindent \textbf{Exercice 8 :} Soit $n$ le nombre de départ. Traduisons les opérations en langage littéral : 
\begin{itemize} 
\item Après multiplication par $2$ : $2n$. 
\item Après addition de $8$ : $2n + 8$. 
\item Prendre la moitié revient à diviser par 2 : $\frac{2n + 8}{2} = n + 4$. 
\item Enfin, soustraire le nombre de départ : $(n + 4) - n = 4$. \end{itemize} 

Après ces quatre étapes, on obtient toujours $\boxed{4}$, quel que soit $n$.
Vérifications numériques : si $n=5$, le résultat est $4$ ; si $n=20$, le résultat est $4$ ; si $n=-10$, on obtient également $4$. Le résultat est \textbf{constant}, ce qui illustre le tour de magie.

\section{Synthèse du chapitre}
\textbf{Ce qu'il faut retenir :} 
\begin{enumerate} 
\item \textbf{Expression littérale :} une expression contenant des lettres (variables) et des nombres. On peut supprimer le signe $\times$ devant une lettre ou une parenthèse. 
\item \textbf{Réduire une expression :} regrouper et calculer les termes semblables pour simplifier l'expression (additionner les coefficients des mêmes variables). 
\item \textbf{Calcul sur une formule :} pour utiliser une formule littérale, on remplace la variable par la valeur donnée et on effectue les opérations. 
\item \textbf{Distributivité simple :} $a(b+c) = ab + ac$. Développer une expression, c'est utiliser cette propriété pour enlever les parenthèses (multiplier chaque terme à l'intérieur par le facteur devant). 
\end{enumerate}
