\documentclass[12pt,a4paper]{book}
\newcommand{\niveau}{6\textsuperscript{e}}
% ================================
% PRÉAMBULE COMMUN — Mathématiques (6e, 5e, 4e)
% ================================

% Encodage et langue
\usepackage[T1]{fontenc}
\usepackage[french]{babel}

% Typographie moderne
\usepackage{newtxtext,newtxmath}

% Configuration des pages
\usepackage{geometry}
\geometry{top=2.5cm, bottom=2.5cm, left=2.5cm, right=2.5cm}

% Amélioration du rendu
\usepackage{microtype}

% Personnalisation des titres
\usepackage{titlesec}
\titleformat{\chapter}[hang]{\huge\bfseries}{\thechapter.}{0.6em}{}
\titleformat{\section}[hang]{\Large\bfseries}{\thesection}{0.6em}{}
\titleformat{\subsection}[hang]{\large\bfseries}{\thesubsection}{0.6em}{}

% Ajustement de l'espacement des titres
\titlespacing{\chapter}{0pt}{-10pt}{20pt}
\titlespacing{\section}{0pt}{12pt}{8pt}
\titlespacing{\subsection}{0pt}{4pt}{2pt}

% En-têtes et pieds de page
\usepackage{fancyhdr}

% Packages mathématiques
\usepackage{amsmath, mathtools}
\usepackage{siunitx}

% Graphiques
\usepackage{tikz}
\usetikzlibrary{angles, quotes, calc, arrows.meta, shapes.geometric}
\usepackage{pgfplots}
\pgfplotsset{compat=1.18}

% Liens hypertexte
\usepackage{hyperref}
\hypersetup{
    colorlinks=true,
    linkcolor=blue!60!black,
    urlcolor=blue!60!black,
    citecolor=blue!60!black
}

% Boîtes colorées avec tcolorbox
\usepackage[most]{tcolorbox}

% Configuration globale des boîtes
\tcbset{
    rounded corners,
    boxsep=2ex,
    top=1.5ex,
    bottom=1.5ex,
    left=2.5ex,
    right=7ex,
    before skip=0pt,
    after skip=1.5ex,
    width=\textwidth,
    boxrule=1pt
}



% Définition des environnements personnalisés
\newtcolorbox{definitionbox}{
    colback=orange!5!white,
    colframe=orange!70!black,
    title={\textbf{Définition}},
    fonttitle=\bfseries,
    coltitle=black
}

\newtcolorbox{examplebox}{
    colback=green!5!white,
    colframe=green!60!black,
    title={\textbf{Exemple}},
    fonttitle=\bfseries,
    coltitle=black
}

\newtcolorbox{exercisebox}{
    colback=purple!5!white,
    colframe=purple!70!black,
    title={\textbf{Exercices}},
    fonttitle=\bfseries,
    coltitle=black
}

\newtcolorbox{objectifsbox}{
    colback=teal!5!white,
    colframe=teal!70!black,
    title={\textbf{Objectifs}},
    fonttitle=\bfseries,
    coltitle=black
}

\newtcolorbox{proprietebox}{
    colback=red!5!white,
    colframe=red!70!black,
    title={\textbf{Propriété}},
    fonttitle=\bfseries,
    coltitle=black
}

\newtcolorbox{activitybox}{
    colback=blue!5!white,
    colframe=blue!70!black,
    title={\textbf{Activité}},
    fonttitle=\bfseries,
    coltitle=black
}

\newtcolorbox{remarkbox}{
    colback=yellow!5!white,
    colframe=yellow!70!black,
    title={\textbf{Remarque}},
    fonttitle=\bfseries,
    coltitle=black
}

\newtcolorbox{quizbox}{
    colback=cyan!5!white,
    colframe=cyan!70!black,
    title={\textbf{Quiz}},
    fonttitle=\bfseries,
    coltitle=black
}

\newtcolorbox{methodebox}{
    colback=purple!5!white,
    colframe=purple!70!black,
    title={\textbf{Méthode}},
    fonttitle=\bfseries,
    coltitle=black
}

% Commande personnalisée pour les trous
\newcommand{\trous}[1]{\makebox[#1]{\rule{0pt}{1.2ex}\dotfill}}

% Variable pour stocker le titre de la séquence
\newcommand{\seqtitle}{}
\newcommand{\setseqtitle}[1]{\renewcommand{\seqtitle}{#1}}

% Mise en page de l'en-tête et du pied
\pagestyle{fancy}
\setlength{\headheight}{16pt} % FIX: évite le warning fancyhdr
\fancyhf{}
\lhead{Mathématiques \niveau{} -- 2025--2026}
\rhead{Seq.~\thechapter~--~\seqtitle}
\cfoot{\thepage}

% Listes compactes
\usepackage{enumitem}
\setlist[itemize]{left=1.2em}
\setlist[enumerate]{left=1.5em}

% Configuration des labels personnalisés pour enumitem
\SetEnumitemKey{a}{label=\alph*)}
\SetEnumitemKey{1}{label=\arabic*)}

% Définition des styles de listes personnalisés
\newlist{exerciselist}{enumerate}{1}
\setlist[exerciselist]{label=\alph*)}

\newlist{quizlist}{enumerate}{1}
\setlist[quizlist]{label=\arabic*)}

% Packages pour tableaux et présentations
\usepackage{longtable}
\usepackage{array}
\usepackage{booktabs}


\title{Cours de Mathématiques — Classe de 6\textsuperscript{e}\\[0.4em]\large Année scolaire 2025–2026}
\author{Abdoullatuf Maoulida}
\date{\today}

\begin{document}
\maketitle
\tableofcontents
\cleardoublepage

%\part{Nombres et calculs}
% Séquence 1 : Les nombres entiers
\setseqtitle{Les nombres entiers}
\chapter{Les nombres entiers}

\section{Rang des chiffres}

\begin{definitionbox}
	\textbf{Chiffres et valeur}
	\begin{itemize}[label = \textbullet]
		\item 0, 1, 2, 3, 4, 5, 6, 7, 8, 9 sont les dix \textbf{chiffres} qui permettent d'écrire tous les nombres.
		\item Chaque chiffre a une \textbf{valeur} en fonction de sa position dans le nombre.
	\end{itemize}
\end{definitionbox}

On peut utiliser un tableau de numération pour visualiser le rang d'un chiffre.

\begin{center}
	\newcolumntype{C}{>{\centering\arraybackslash}p{0.8cm}}
	\begin{tabular}{|C|C|C|C|C|C|C|C|C|C|C|C|}
		\hline
		\multicolumn{3}{|c|}{\textbf{Milliards}} & 
		\multicolumn{3}{|c|}{\textbf{Millions}} & 
		\multicolumn{3}{|c|}{\textbf{Milliers}} & 
		\multicolumn{3}{|c|}{\textbf{Unités}} \\
		\hline
		\textbf{c} & \textbf{d} & \textbf{u} & 
		\textbf{c} & \textbf{d} & \textbf{u} & 
		\textbf{c} & \textbf{d} & \textbf{u} & 
		\textbf{c} & \textbf{d} & \textbf{u} \\
		\hline
		& & & & & & & & & & & \\
		\hline
	\end{tabular}
\end{center}

\textbf{Remarque :} Lorsqu'on écrit un nombre en chiffres, il faut laisser un espace entre les classes. Par exemple le nombre suivant 25204879603 s'écrit: \trous{2.5cm}.

\section{Décomposition décimale}

On peut donner la décomposition décimale de 3 584 :

\begin{examplebox}
	3 584 = (\trous{1cm} $\times$ \trous{2cm}) + (\trous{1cm} $\times$ \trous{1.5cm}) + (\trous{0.5cm} $\times$ \trous{1cm}) + \trous{0.5cm} $\times$ \trous{0.5cm}
\end{examplebox}



\textbf{Attention !} Pour le nombre 3 584, le \textbf{chiffre} des centaines est \trous{2cm} mais le \textbf{nombre} de centaines est \trous{3cm} (il y a \trous{2cm} centaines dans le nombre 3584).

En effet :
\trous{14cm}

\trous{15.5cm}


\begin{examplebox}
Dans le nombre 25 803,\\ 

le chiffre des dizaines est \trous{1.5cm}; le nombre de dizaines est \trous{2.5cm}\\

le chiffre des centaines est \trous{1.5cm} ; le nombre de centaines est \trous{2.5cm}
\end{examplebox}

\section{Écriture en toutes lettres}

\begin{examplebox}
	\begin{itemize}[label = \textbullet]
		\item 1823 : Mille-huit-cent-vingt-trois (pas de << s >> à << cent >>, ni à << vingt >> car ils sont suivis d'autres chiffres !)
		\item 2087 : Deux-mille-quatre-vingt-sept (le mot << mille >> est invariable, et toujours pas de << s >> à << vingt >>...)
		\item 600 : Six-cents (ici on met bien un << s >> car il n'y a plus rien derrière !)
		\item 680 : Six-cent-quatre-vingts (pas de << s >> à << cent >>, mais un << s >> obligatoire à << vingt >> car le nombre se termine par 80).
	\end{itemize}
\end{examplebox}

Voici les règles correspondantes à ces exemples :

\begin{itemize}
	\item Le mot << mille >> est invariable ; les mots << million >> et << milliard >>, cependant, s'accordent et prennent donc un << \textbf{s} >> au pluriel.
	\item Les mots << cent >> et << vingt >> prennent un << \textbf{s} >> au pluriel uniquement lorsqu'ils sont à la fin du nombre.
	\item \textbf{Exemples :} 300 : \trous{5cm} \qquad 420 : \trous{5cm}
	\item Le mot << vingt >> ne s'utilise au pluriel (avec un << s >>) que si un nombre se finit par 80 (quatre-vingts).
	\item Les tirets sont mis entre chaque mot d'un nombre qui se présente sous forme composée. Avec des nombres entiers, il y aura donc des tirets partout !
	\item \textbf{Exemples :} \\
	79 : \trous{6cm} \\
	1031 : \trous{6cm}
\end{itemize}

\section{Demi-droite graduée}

\begin{definitionbox}
	\textbf{Demi-droite graduée}
	
	On appelle \textbf{demi-droite graduée} une demi-droite sur laquelle on fixe :
	\begin{itemize}[label = \textbullet]
		\item Un point appelé \textbf{origine de la demi-droite}
		\item \textbf{Un sens} représenté par une flèche
		\item \textbf{Une unité de longueur} que l'on reporte régulièrement à partir de l'origine.
	\end{itemize}
\end{definitionbox}

\begin{figure}[h]
	\centering
	\includegraphics[width=0.6\linewidth]{../../assets/images/6e/seq_01/demi-droite-graduee.png}
	\caption{Demi-droite graduée}
	\label{fig:demi-droite-graduee}
\end{figure}

\begin{proprietebox}
	Sur une demi-droite graduée,
	\begin{itemize}[label = \textbullet]
	 \item Chaque point est repéré par \trous{3cm} appelé \trous{3cm} de ce point.
	 \item A chaque nombre correspond \trous{3cm} unique.
	\end{itemize}
\end{proprietebox}

\begin{examplebox}
	% Schéma de droite graduée avec TikZ
	\begin{center}
		\begin{tikzpicture}[scale=0.8]
			% Droite graduée
			\draw[thick, ->] (0,0) -- (12,0);
			
			% Graduations
			\foreach \x in {0,1,...,11} {
				\draw (\x,-0.1) -- (\x,0.1);
				\node[below] at (\x,-0.3) {\x};
			}
			
			% Point A
			\node[circle,fill=black,inner sep=2pt] at (3.5,0) {};
			\node[above] at (3.5,0.2) {$A$};
			
			% Point B
			\node[circle,fill=black,inner sep=2pt] at (7,0) {};
			\node[above] at (7.2,0.2) {$B$};
		\end{tikzpicture}
	\end{center}
	
	Sur cette demi-droite graduée, le point $A$ a pour abscisse \trous{1.5cm} et le point $B$ a pour abscisse \trous{1.5cm}.	
\end{examplebox}


\begin{center}
	\includegraphics[width=1\linewidth]{../../assets/images/6e/seq_01/attention-demi-dte-graduee}
\end{center}


\section{Exercices d'application}

\begin{exercisebox}
\textbf{Exercice 1 :} Écris en toutes lettres les nombres suivants :
\begin{enumerate}
	\item 1 234
	\item 5 678
	\item 12 345
	\item 100 000
\end{enumerate}

\textbf{Exercice 2 :} Place les points $A$, $B$, $C$ et $D$ d'abscisses respectives 2, 7, 4 et 9 sur une demi-droite graduée.


\end{exercisebox}



 % Les nombres entiers
% % Séquence 2 : Points et droites
\setseqtitle{Points et droites}
\chapter{Points et droites}

\begin{objectifsbox}
\textbf{Objectifs d'apprentissage.} À l'issue de la séquence, l'élève sera capable de :
\begin{itemize}
\item Représenter des points et des droites
\item Utiliser le vocabulaire géométrique approprié
\item Construire des figures géométriques simples
\end{itemize}
\end{objectifsbox}

\section{Représentation des points}

\begin{definitionbox}
Un \textbf{point} est représenté par une croix ou un petit cercle plein.
\end{definitionbox}

\begin{examplebox}
\textbf{Exemple :} Le point A se note A et se représente par $\times$ ou $\bullet$
\end{examplebox}

\section{Représentation des droites}

\begin{definitionbox}
Une \textbf{droite} est une ligne droite qui s'étend à l'infini dans les deux sens.
\end{definitionbox}

\begin{proprietebox}
\textbf{Propriété :} Par deux points distincts, il passe une seule droite.
\end{proprietebox}

\begin{exercisebox}
\textbf{Exercice d'application :}
Tracer la droite passant par les points A et B.
\end{exercisebox}
 % Points et droites
% % Séquence 3 : Fractions décimales et nombres décimaux
\setseqtitle{Fractions décimales et nombres décimaux}
\chapter{Fractions décimales et nombres décimaux}

\begin{objectifsbox}
	\textbf{Objectifs d'apprentissage de la séquence}
	\begin{itemize}
		\item Reconnaître un nombre décimal
		\item Connaître les liens entre les unités de numération unité, dizaine, centaine, millier, dixième, centième, millième
		\item Associer et utiliser différentes écritures d'un nombre décimal : écriture à virgule, fraction, nombre mixte, pourcentage
		\item Comparer deux nombres décimaux
		\item Ordonner une liste de nombres décimaux
		\item Encadrer un nombre décimal par deux nombres décimaux, intercaler un nombre décimal entre deux nombres décimaux
		\item Placer sur une demi-droite graduée un point dont l'abscisse est un nombre décimal
		\item Repérer un nombre décimal sur une demi-droite graduée
	\end{itemize}
\end{objectifsbox}

\section{Définitions et exemples}

\begin{definitionbox}
	\textbf{Nombre décimal et fraction décimale}
	
	Un \textbf{nombre décimal} est un nombre qui peut s'écrire avec une virgule et qui possède un nombre fini de chiffres après la virgule.
	
	Une \textbf{fraction décimale} est une fraction dont le dénominateur est 10, 100, 1000, etc.
\end{definitionbox}

\begin{examplebox}
	\begin{itemize}
		\item 0,1 ; 0,7 ; 0,01 et 0,001 sont des nombres décimaux
		\item $\frac{1}{10}$ ; $\frac{7}{10}$ ; $\frac{1}{100}$ et $\frac{1}{1000}$ sont des fractions décimales
		\item 12,56 = $\frac{1256}{100}$ et 0,025 = $\frac{25}{1000}$
	\end{itemize}
\end{examplebox}

\textbf{Correspondances importantes :}
\begin{center}
	\begin{tabular}{ccc}
		Un dixième & Sept dixièmes & Un centième \\
		$\frac{1}{10}$ = \trous{1.5cm} & $\frac{7}{10}$ = \trous{1.5cm} & $\frac{1}{100}$ = \trous{1.5cm} \\
		  
		\trous{2.5cm} & \trous{2.5cm} & \trous{2.5cm} \\
	\end{tabular}
\end{center}

\section{Décompositions d'un nombre décimal}

On peut représenter un nombre décimal dans un tableau de numération :

\begin{center}
	\begin{tabular}{|c|c|c|c|c|c|c|}
		\hline
		\multicolumn{2}{|c|}{\textbf{Partie entière}} & & \multicolumn{4}{|c|}{\textbf{Partie décimale}} \\
		\hline
		Dizaines & Unités & , & Dixièmes & Centièmes & Millièmes & Dix-millièmes \\
		\hline
		1 & 5 & , & 9 & 3 & 1 & \\
		\hline
	\end{tabular}
\end{center}

\textbf{Exemple :} Pour le nombre 15,931 :

\begin{itemize}
	\item Décomposition additive : 15,931 = \trous{1cm} + $\frac{\trous{1cm}}{10}$ + $\frac{\trous{1cm}}{100}$ + $\frac{\trous{1cm}}{1000}$
	\item Décomposition simplifiée : 15,931 = \trous{1cm} + $\frac{\trous{3cm}}{1000}$
	\item Écriture fractionnaire : 15,931 = $\frac{\trous{4cm}}{1000}$
\end{itemize}

\textbf{Différentes lectures possibles du nombre 15,931 :}
\begin{itemize}
	\item << \trous{8cm} >>
	\item << \trous{8cm} >>
	\item << \trous{8cm} >>
	\item << \trous{8cm} >>
\end{itemize}

\section{Comparer deux nombres décimaux}

\begin{definitionbox}
	\textbf{Comparer deux nombres}
	
	Comparer deux nombres, c'est dire lequel est le plus grand ou s'ils sont égaux.
	
	On utilise les symboles : $<$ (plus petit que), $>$ (plus grand que), $=$ (égal à).
\end{definitionbox}

\textbf{Méthode pour comparer deux nombres décimaux :}

\begin{examplebox}
	\textbf{Exemple 1 :} Comparer 14,12 et 11,865.
	
	On commence par comparer \trous{4cm} :
	si elles sont différentes, le nombre qui a la plus grande partie entière est le plus grand.
	
	14,12 > 11,865 se lit << \trous{6cm} >>
	11,865 < 14,12 se lit << \trous{6cm} >>
\end{examplebox}

\begin{examplebox}
	\textbf{Exemple 2 :} Comparer 27,28 et 27,6.
	
	Si les deux nombres ont la même partie entière, on compare \trous{4cm} : s'ils sont différents, le nombre qui a le plus grand chiffre des dixièmes est le plus grand.
	
	Il s'agit ici de comparer les parties décimales des deux nombres :
	27,28 < 27,6 car \trous{6cm}
\end{examplebox}

\begin{examplebox}
	\textbf{Exemple 3 :} Comparer 8,0171 et 8,0159.
	
	Si les deux nombres ont le même chiffre des dixièmes, on fait de même avec les centièmes, les millièmes...
	
	8,0171 > 8,0159 car \trous{6cm}
\end{examplebox}

\section{Ranger des nombres décimaux}

\begin{definitionbox}
	\textbf{Ranger des nombres}
	
	Ranger des nombres dans l'ordre croissant, c'est les ranger \trous{4cm}.
	
	Ranger des nombres dans l'ordre décroissant, c'est les ranger \trous{4cm}.
\end{definitionbox}

\begin{examplebox}
	\textbf{Ranger dans l'ordre décroissant} les nombres : 15,78 ; 15,751 ; 16,01 ; 15,8 ; 16,1
	
	On commence par chercher le plus grand nombre : \trous{3cm}
	
	\trous{15cm}
	
	Quand on range des nombres dans l'ordre décroissant, on les sépare par le symbole << > >>.
\end{examplebox}

\begin{examplebox}
	\textbf{Ranger dans l'ordre croissant} les nombres : 3,25 ; 2,36 ; 3,205 ; 3,3 ; 2,29
	
	On commence par chercher le plus petit nombre : \trous{3cm}
	
	\trous{15cm}
	
	Quand on range des nombres dans l'ordre croissant, on les sépare par le symbole << < >>.
\end{examplebox}

\section{Encadrer un nombre décimal}

\begin{definitionbox}
	\textbf{Encadrer un nombre}
	
	Encadrer un nombre, c'est trouver deux nombres, l'un plus petit et l'autre plus grand, entre lesquels se situe ce nombre.
\end{definitionbox}

\begin{examplebox}
	\textbf{Exemples d'encadrements :}
	\begin{itemize}
		\item Donner un encadrement à l'unité de 12,27 : \trous{4cm} < 12,27 < \trous{4cm}
		
		On lit << \trous{5cm} >>
		
		On veut encadrer 12,27 entre deux nombres dont la différence est une unité.
		
		D'autres réponses sont justes : \trous{6cm}
		
		\item Donner un encadrement au dixième de 3,526 : \trous{4cm} < 3,526 < \trous{4cm}
		
		On lit << \trous{5cm} >>
		
		On veut encadrer 3,526 entre deux nombres dont la différence est un dixième.
		
		\item Donner un encadrement au centième de 1,159 : \trous{4cm} < 1,159 < \trous{4cm}
		
		\item Donner un encadrement au millième de 7,1459 : \trous{4cm} < 7,1459 < \trous{4cm}
	\end{itemize}
\end{examplebox}

\section{Lire l'abscisse décimale d'un point}

\textbf{Méthode :} Pour lire l'abscisse d'un point sur une demi-droite graduée, il faut :
\begin{enumerate}
	\item Identifier l'unité et voir en combien de parts elle est divisée
	\item Calculer la valeur de chaque graduation
	\item Compter les graduations depuis l'origine
\end{enumerate}

\begin{examplebox}
	\textbf{Exemple 1 :}
	
	% Schéma avec points à compléter
	\begin{center}
		\begin{tikzpicture}[scale=1.2]
			% Droite graduée de 3 à 4
			\draw[thick, ->] (0,0) -- (10,0);
			
			% Graduations principales
			\foreach \x in {0,10} {
				\draw (\x,-0.2) -- (\x,0.2);
				\node[below] at (\x,-0.4) {\ifnum\x=0 3\else 4\fi};
			}
			
			% Subdivisions (10 intervalles)
			\foreach \x in {1,2,...,9} {
				\draw (\x,-0.1) -- (\x,0.1);
			}
			
			% Points exemple
			\fill[blue] (7,0) circle (0.1);
			\node[above] at (7,0.3) {A};
			
			\fill[red] (2,0) circle (0.1);
			\node[above] at (2,0.3) {B};
			
			\fill[green] (9,0) circle (0.1);
			\node[above] at (9,0.3) {C};
		\end{tikzpicture}
	\end{center}
	
	L'unité a été partagée en \trous{2cm} intervalles : chaque intervalle mesure donc \trous{3cm} de longueur.
	
	L'abscisse du point A est \trous{2cm}
	
	L'abscisse du point B est \trous{2cm}
	
	L'abscisse du point C est \trous{2cm}
	
	On peut noter : A(\trous{2cm}), B(\trous{2cm}), C(\trous{2cm})
	
	1 ÷ 10 = \trous{2cm}
\end{examplebox}

\begin{examplebox}
	\textbf{Exemple 2 :}
	
	% Schéma avec points entre 5,6 et 5,7
	\begin{center}
		\begin{tikzpicture}[scale=1.2]
			% Droite graduée de 5,6 à 5,7
			\draw[thick, ->] (0,0) -- (10,0);
			
			% Graduations principales
			\foreach \x in {0,10} {
				\draw (\x,-0.2) -- (\x,0.2);
				\node[below] at (\x,-0.4) {\ifnum\x=0 5,6\else 5,7\fi};
			}
			
			% Subdivisions (10 intervalles)
			\foreach \x in {1,2,...,9} {
				\draw (\x,-0.1) -- (\x,0.1);
			}
			
			% Points exemple
			\fill[blue] (6,0) circle (0.1);
			\node[above] at (6,0.3) {D};
			
			\fill[red] (1,0) circle (0.1);
			\node[above] at (1,0.3) {E};
			
			\fill[green] (8,0) circle (0.1);
			\node[above] at (8,0.3) {F};
		\end{tikzpicture}
	\end{center}
	
	Un dixième d'unité a été partagé en \trous{2cm} intervalles : chaque intervalle mesure donc \trous{3cm} de longueur.
	
	L'abscisse du point D est \trous{2cm}
	
	L'abscisse du point E est \trous{2cm}
	
	L'abscisse du point F est \trous{2cm}
	
	On peut noter : D(\trous{2cm}), E(\trous{2cm}), F(\trous{2cm})
	
	0,1 ÷ 10 = \trous{2cm}
\end{examplebox}

\section{Lire l'abscisse d'un point par agrandissements successifs}

\textbf{Méthode des agrandissements successifs :}

Cette méthode consiste à << zoomer >> progressivement sur la partie de la droite graduée qui nous intéresse pour lire une abscisse avec plus de précision.

\begin{examplebox}
	% Schémas d'agrandissements successifs à compléter
	\textbf{Étape 1 :} Une centaine a été partagée en 10 : chaque graduation correspond donc à \trous{4cm}
	
	L'abscisse du point A est comprise entre \trous{3cm} et \trous{3cm}
	
	\textbf{Étape 2 :} Une dizaine a été partagée en 10 : chaque graduation correspond donc à \trous{4cm}
	
	L'abscisse du point A est comprise entre \trous{3cm} et \trous{3cm}
	
	\textbf{Étape 3 :} Une unité a été partagée en 10 : chaque graduation correspond donc à \trous{4cm}
	
	L'abscisse du point A est comprise entre \trous{3cm} et \trous{3cm}
	
	\textbf{Étape 4 :} Un dixième a été partagé en 10 : chaque graduation correspond donc à \trous{4cm}
	
	L'abscisse du point A est \trous{3cm}
\end{examplebox}

\vspace{2cm}

\textbf{Exercices d'application :}

\begin{exercisebox}
	1. Écrire sous forme de fraction décimale puis sous forme décimale :
	\begin{itemize}
		\item Trois dixièmes : \trous{3cm} = \trous{2cm}
		\item Vingt-sept centièmes : \trous{3cm} = \trous{2cm}
		\item Cent quarante-cinq millièmes : \trous{3cm} = \trous{2cm}
	\end{itemize}
	
	2. Comparer les nombres suivants (utiliser les symboles <, > ou =) :
	\begin{itemize}
		\item 12,3 \trous{1cm} 12,30
		\item 5,67 \trous{1cm} 5,7
		\item 8,09 \trous{1cm} 8,1
	\end{itemize}
	
	3. Ranger dans l'ordre croissant : 2,1 ; 2,01 ; 2,11 ; 2,101
	
	\trous{8cm}
	
	4. Donner un encadrement au dixième de 7,384 :
	
	\trous{5cm} < 7,384 < \trous{5cm}
\end{exercisebox} % Fractions décimales et nombres décimaux
% % Séquence 4 : Distance, cercle et triangles
\setseqtitle{Distance, cercle et triangles}
\chapter{Distance, cercle et triangles}

\begin{objectifsbox}
\textbf{Objectifs d'apprentissage.} À l'issue de la séquence, l'élève sera capable de :
\begin{itemize}
\item Calculer des distances
\item Construire des cercles
\item Reconnaître différents types de triangles
\end{itemize}
\end{objectifsbox}

\section{La distance}

\begin{definitionbox}
La \textbf{distance} entre deux points est la longueur du segment qui les relie.
\end{definitionbox}

\begin{examplebox}
\textbf{Exemple :} La distance entre les points A et B se note AB.
\end{examplebox}

\section{Le cercle}

\begin{definitionbox}
Un \textbf{cercle} est l'ensemble des points situés à la même distance d'un point appelé centre.
\end{definitionbox}

\begin{proprietebox}
\textbf{Propriété :} Tous les points du cercle sont à égale distance du centre.
\end{proprietebox}

\begin{exercisebox}
\textbf{Exercice d'application :}
Tracer un cercle de centre O et de rayon 3 cm.
\end{exercisebox}
 % Distance, cercle et triangles
% \input{chapitres/seq_05} % Notion de proportionnalité
% \input{chapitres/seq_06} % Notion de probabilités
% \input{chapitres/seq_07} % Angles et rapporteur
% \setseqtitle{Opérations avec les nombres décimaux}
\chapter{Opérations avec les nombres décimaux}
\label{chap:seq08}

\begin{objectifsbox}
\textbf{Objectifs d'apprentissage.} À l'issue de la séquence, l'élève sera capable de :
\begin{itemize}
  \item Additionner et soustraire des nombres décimaux
  \item Multiplier des nombres décimaux
  \item Poser correctement les opérations avec des nombres décimaux
  \item Calculer des ordres de grandeur
\end{itemize}
\end{objectifsbox}

\section*{Pré-requis}
\begin{itemize}
  \item Connaissance des nombres entiers et de leurs opérations
  \item Maîtrise de la numération décimale
  \item Compréhension de la valeur des chiffres selon leur position
\end{itemize}

\section{Addition et soustraction avec des nombres décimaux}

\begin{definitionbox}
	\textbf{Addition et soustraction}
	
	On calcule une \textbf{\trous{3cm}} lorsqu'on ajoute deux nombres, et une \textbf{\trous{3cm}} lorsqu'on en soustrait deux.
	
	Le résultat d'une addition est une \textbf{\trous{2.5cm}}, celui d'une soustraction une \textbf{\trous{2.5cm}}.
	
	Les nombres calculés ensemble s'appellent les \textbf{\trous{2.5cm}}.
\end{definitionbox}

\begin{examplebox}
	\begin{center}
		\begin{tabular}{p{0.45\textwidth}p{0.45\textwidth}}
			\textbf{21,5 + 12,3 = 33,8} (\trous{2.5cm}) & \textbf{21,5 - 12,3 = 9,2} (\trous{2.5cm}) \\
			On dit que << 33,8 est la somme de 21,5 et 12,3 >> & On dit que << 9,2 est la différence de 21,5 par 12,3 >> \\
		\end{tabular}
	\end{center}
	
	Dans les deux cas, les deux nombres \textbf{21,5} et \textbf{12,3} sont les \textbf{termes} du calcul.
\end{examplebox}

\begin{proprietebox}
On peut échanger les termes d'une addition sans modifier son résultat. On dit que l'addition est \textbf{\trous{3cm}}.
\end{proprietebox}

\textbf{ATTENTION :} Ce n'est pas vrai pour une soustraction !

% Exemple 1 : opération en ligne dans une boxe
\begin{examplebox}
	\begin{center}
		\begin{tabular}{p{0.45\textwidth} p{0.45\textwidth}}
			\textbf{Exemple 1 (opération en ligne)} : \\
			$8,5 + 7,2 + 2,1 + 3,4$ & $8,5 - 3,2$ \\
			$= \underline{\trous{1cm} + \trous{1cm}} + \underline{\trous{1cm} + \trous{1cm}}$ & $= 5,3$ \\
			$= 10,6 + 10,6$ & (attention : on ne sait pas encore calculer $3,2 - 8,5$) \\
			$= 21,2$ & \\
		\end{tabular}
	\end{center}
\end{examplebox}

\vspace{0.5cm}

% Exemple 2 : opération posée dans une boxe
\begin{examplebox}
	\begin{center}
		\begin{tabular}{p{0.45\textwidth} p{0.45\textwidth}}
			\textbf{Exemple 2 (opération posée)} : \underline{\textbf{28,4 + 84,39}} 
			& 
			\underline{\textbf{20,18 - 19,45}} \\
			
			\begin{tabular}{r}
				84,39 \\
				+ 28,40 \\
				\hline
				\trous{1.5cm}
			\end{tabular} &
			\begin{tabular}{r}
				20,18 \\
				- 19,45 \\
				\hline
				\trous{1.5cm}
			\end{tabular} \\
			
		\end{tabular}
	\end{center}
\end{examplebox}


\textbf{Remarques :}
\begin{itemize}
	\item Les mots << \textbf{addition} >> et << \textbf{soustraction} >> désignent des opérations, tandis que les mots << \textbf{somme} >> et << \textbf{différence} >> désignent des nombres (des résultats).
	\item Pour poser une addition ou une soustraction de nombres décimaux, il faut impérativement aligner les nombres par la droite et aligner les virgules.
	\item On peut ajouter des zéros à droite d'un nombre décimal sans changer sa valeur (exemple : 28,4 = 28,40).
\end{itemize}

\section{Multiplication avec des nombres décimaux}

\begin{definitionbox}
	\textbf{Multiplication}
	
	Dans une \textbf{multiplication}, on multiplie des \trous{2.5cm}, et le résultat est un \trous{2cm}.
\end{definitionbox}

\begin{examplebox}
	On dit que 60,5 est le \trous{2.5cm} de 12,1 par 5
\end{examplebox}

\begin{proprietebox}
On peut échanger l'ordre des facteurs sans changer le résultat. On dit que la multiplication est \trous{3cm}.
\end{proprietebox}

\textbf{Méthode :} Poser une multiplication avec des nombres décimaux.

\textbf{Exemple :} $25,1 \times 7,53$

\begin{center}
	\begin{tabular}{r}
		25,1 \\
		$\times$ 7,53 \\
		\hline
		\trous{2cm} \\
		\trous{2cm} \\
		\trous{2cm} \\
		\hline
		\trous{2cm}
	\end{tabular}
\end{center}

\textbf{Règle pour placer la virgule :} Le nombre de chiffres après la virgule dans le résultat est égal à la somme du nombre de chiffres après la virgule dans chaque facteur.

\section{Ordre de grandeur}

\begin{definitionbox}
	\textbf{Ordre de grandeur}
	
	Un \textbf{ordre de grandeur} d'un nombre est un nombre proche de celui-ci et facile à utiliser en calcul mental.
	
	\textbf{Remarque :} Un ordre de grandeur n'est pas unique : on peut donner des ordres de grandeur différents selon la précision voulue.
\end{definitionbox}

\begin{examplebox}
	La population française était de 67 063 703 habitants en 2020. Un ordre de grandeur de cette population est \trous{3cm} (on pourrait aussi choisir 100 millions ou 67 millions).
\end{examplebox}

\begin{examplebox}
Pour calculer $24,7 \times 3,8$, on peut d'abord estimer :

• 24,7 $\approx$ 25 (ordre de grandeur)

• 3,8 $\approx$ 4 (ordre de grandeur)

• Donc $24,7 \times 3,8 \approx 25 \times 4 = 100$

Le résultat exact sera proche de 100.
\end{examplebox}

\section{Exercices d'entraînement}

\begin{exercisebox}
\textbf{Exercices d'application :} Pose et calcule les opérations suivantes :
\begin{enumerate}
	\item 45,7 + 23,8
	\item 67,2 - 34,5
	\item $12,3 \times 4,6$
\end{enumerate}
\end{exercisebox}

\begin{exercisebox}
\textbf{Exercices d'application :} Donne un ordre de grandeur de chaque calcul, puis calcule le résultat exact :
\begin{enumerate}
	\item 23,4 + 45,7
	\item 89,2 - 12,8
	\item $15,6 \times 3,2$
\end{enumerate}
\end{exercisebox}

\section{Évaluation rapide (5 à 10 min)}

\begin{exercisebox}
\textbf{Évaluation rapide :}
\begin{enumerate}
	\item Quel est le résultat de 12,5 + 8,7 ?
	\item Calcule 45,2 - 23,8
	\item Donne un ordre de grandeur de $34,7 \times 2,1$
\end{enumerate}
\end{exercisebox}
 % Opérations avec les nombres décimaux
% \input{chapitres/seq_09} % La médiatrice d'un segment
% \input{chapitres/seq_10} % La division
% \chapter{Symétrie axiale}
\label{chap:seq11}

\begin{definitionbox}
\textbf{Objectifs d'apprentissage.} À l'issue de la séquence, l'élève sera capable de :
\begin{itemize}
  \item ...
\end{itemize}
\end{definitionbox}

\section*{Pré-requis}
\begin{itemize}
  \item ...
\end{itemize}

\section{Découverte}
\begin{examplebox}
Problème d'introduction ou situation de découverte.
\end{examplebox}

\section{Leçon}
\begin{itemize}
  \item Rappels et définitions.
  \item Méthodes et exemples guidés.
\end{itemize}

\section{Exercices d'entraînement}
\begin{exercisebox}
Exercice 1.
\end{exercisebox}

\begin{exercisebox}
Exercice 2.
\end{exercisebox}

\section{Évaluation rapide (5 à 10 min)}
\begin{exercisebox}
Mini-quiz.
\end{exercisebox}
 % Symétrie axiale
% \input{chapitres/seq_12} % Fraction partage et comparaison de fractions
% \input{chapitres/seq_13} % Unités de longueur, de masse et de contenance
% \input{chapitres/seq_14} % Calculer avec les angles
% \input{chapitres/seq_15} % Nombres en écriture fractionnaire
% \input{chapitres/seq_16} % Proportionnalité et pourcentages
% \input{chapitres/seq_17} % Déterminer des probabilités et des issues
% \input{chapitres/seq_18} % Aires et périmètres
% \input{chapitres/seq_19} % Heures et durées
% \input{chapitres/seq_20} % Solides et volumes

\cleardoublepage
\appendix
\chapter{Progression annuelle (récapitulatif)}
Cette progression correspond à la répartition établie pour l'année 2025–2026.

\begin{center}
\begin{tabular}{|l|l|}
\hline
\textbf{Période} & \textbf{Séquences}\\ \hline
Période 1 (6 semaines) & S01 -- Les nombres entiers, S02 -- Points et droites, S03 -- Fractions décimales et nombres décimaux\\ \hline
Période 2 (7 semaines) & S04 -- Distance, cercle et triangles, S05 -- Notion de proportionnalité, S06 -- Notion de probabilités, S07 -- Angles et rapporteur\\ \hline
Période 3 (6 semaines) & S08 -- Opérations avec les nombres décimaux, S09 -- La médiatrice d'un segment, S10 -- La division, S11 -- Symétrie axiale\\ \hline
Période 4 (7 semaines) & S12 -- Fraction partage et comparaison de fractions, S13 -- Unités de longueur, de masse et de contenance, S14 -- Calculer avec les angles, S15 -- Nombres en écriture fractionnaire\\ \hline
Période 5 (6 semaines) & S16 -- Proportionnalité et pourcentages, S17 -- Déterminer des probabilités et des issues, S18 -- Aires et périmètres, S19 -- Heures et durées, S20 -- Solides et volumes\\ \hline
\end{tabular}
\end{center}

\end{document}
