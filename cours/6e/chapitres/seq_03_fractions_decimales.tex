% Séquence 3 : Fractions décimales et nombres décimaux
\setseqtitle{Fractions décimales et nombres décimaux}
\chapter{Fractions décimales et nombres décimaux}

\begin{objectifsbox}
	\textbf{Objectifs d'apprentissage de la séquence}
	\begin{itemize}
		\item Reconnaître un nombre décimal
		\item Connaître les liens entre les unités de numération unité, dizaine, centaine, millier, dixième, centième, millième
		\item Associer et utiliser différentes écritures d'un nombre décimal : écriture à virgule, fraction, nombre mixte, pourcentage
		\item Comparer deux nombres décimaux
		\item Ordonner une liste de nombres décimaux
		\item Encadrer un nombre décimal par deux nombres décimaux, intercaler un nombre décimal entre deux nombres décimaux
		\item Placer sur une demi-droite graduée un point dont l'abscisse est un nombre décimal
		\item Repérer un nombre décimal sur une demi-droite graduée
	\end{itemize}
\end{objectifsbox}

\section{Définitions et exemples}

\begin{definitionbox}
	\textbf{Nombre décimal et fraction décimale}
	
	Un \textbf{nombre décimal} est un nombre qui peut s'écrire avec une virgule et qui possède un nombre fini de chiffres après la virgule.
	
	Une \textbf{fraction décimale} est une fraction dont le dénominateur est 10, 100, 1000, etc.
\end{definitionbox}

\begin{examplebox}
	\begin{itemize}
		\item 0,1 ; 0,7 ; 0,01 et 0,001 sont des nombres décimaux
		\item $\frac{1}{10}$ ; $\frac{7}{10}$ ; $\frac{1}{100}$ et $\frac{1}{1000}$ sont des fractions décimales
		\item 12,56 = $\frac{1256}{100}$ et 0,025 = $\frac{25}{1000}$
	\end{itemize}
\end{examplebox}

\textbf{Correspondances importantes :}
\begin{center}
	\begin{tabular}{ccc}
		Un dixième & Sept dixièmes & Un centième \\
		$\frac{1}{10}$ = \trous{1.5cm} & $\frac{7}{10}$ = \trous{1.5cm} & $\frac{1}{100}$ = \trous{1.5cm} \\
		  
		\trous{2.5cm} & \trous{2.5cm} & \trous{2.5cm} \\
	\end{tabular}
\end{center}

\section{Décompositions d'un nombre décimal}

On peut représenter un nombre décimal dans un tableau de numération :

\begin{center}
	\begin{tabular}{|c|c|c|c|c|c|c|}
		\hline
		\multicolumn{2}{|c|}{\textbf{Partie entière}} & & \multicolumn{4}{|c|}{\textbf{Partie décimale}} \\
		\hline
		Dizaines & Unités & , & Dixièmes & Centièmes & Millièmes & Dix-millièmes \\
		\hline
		1 & 5 & , & 9 & 3 & 1 & \\
		\hline
	\end{tabular}
\end{center}

\textbf{Exemple :} Pour le nombre 15,931 :

\begin{itemize}
	\item Décomposition additive : 15,931 = \trous{1cm} + $\frac{\trous{1cm}}{10}$ + $\frac{\trous{1cm}}{100}$ + $\frac{\trous{1cm}}{1000}$
	\item Décomposition simplifiée : 15,931 = \trous{1cm} + $\frac{\trous{3cm}}{1000}$
	\item Écriture fractionnaire : 15,931 = $\frac{\trous{4cm}}{1000}$
\end{itemize}

\textbf{Différentes lectures possibles du nombre 15,931 :}
\begin{itemize}
	\item << \trous{8cm} >>
	\item << \trous{8cm} >>
	\item << \trous{8cm} >>
	\item << \trous{8cm} >>
\end{itemize}

\section{Comparer deux nombres décimaux}

\begin{definitionbox}
	\textbf{Comparer deux nombres}
	
	Comparer deux nombres, c'est dire lequel est le plus grand ou s'ils sont égaux.
	
	On utilise les symboles : $<$ (plus petit que), $>$ (plus grand que), $=$ (égal à).
\end{definitionbox}

\textbf{Méthode pour comparer deux nombres décimaux :}

\begin{examplebox}
	\textbf{Exemple 1 :} Comparer 14,12 et 11,865.
	
	On commence par comparer \trous{4cm} :
	si elles sont différentes, le nombre qui a la plus grande partie entière est le plus grand.
	
	14,12 > 11,865 se lit << \trous{6cm} >>
	11,865 < 14,12 se lit << \trous{6cm} >>
\end{examplebox}

\begin{examplebox}
	\textbf{Exemple 2 :} Comparer 27,28 et 27,6.
	
	Si les deux nombres ont la même partie entière, on compare \trous{4cm} : s'ils sont différents, le nombre qui a le plus grand chiffre des dixièmes est le plus grand.
	
	Il s'agit ici de comparer les parties décimales des deux nombres :
	27,28 < 27,6 car \trous{6cm}
\end{examplebox}

\begin{examplebox}
	\textbf{Exemple 3 :} Comparer 8,0171 et 8,0159.
	
	Si les deux nombres ont le même chiffre des dixièmes, on fait de même avec les centièmes, les millièmes...
	
	8,0171 > 8,0159 car \trous{6cm}
\end{examplebox}

\section{Ranger des nombres décimaux}

\begin{definitionbox}
	\textbf{Ranger des nombres}
	
	Ranger des nombres dans l'ordre croissant, c'est les ranger \trous{4cm}.
	
	Ranger des nombres dans l'ordre décroissant, c'est les ranger \trous{4cm}.
\end{definitionbox}

\begin{examplebox}
	\textbf{Ranger dans l'ordre décroissant} les nombres : 15,78 ; 15,751 ; 16,01 ; 15,8 ; 16,1
	
	On commence par chercher le plus grand nombre : \trous{3cm}
	
	\trous{15cm}
	
	Quand on range des nombres dans l'ordre décroissant, on les sépare par le symbole << > >>.
\end{examplebox}

\begin{examplebox}
	\textbf{Ranger dans l'ordre croissant} les nombres : 3,25 ; 2,36 ; 3,205 ; 3,3 ; 2,29
	
	On commence par chercher le plus petit nombre : \trous{3cm}
	
	\trous{15cm}
	
	Quand on range des nombres dans l'ordre croissant, on les sépare par le symbole << < >>.
\end{examplebox}

\section{Encadrer un nombre décimal}

\begin{definitionbox}
	\textbf{Encadrer un nombre}
	
	Encadrer un nombre, c'est trouver deux nombres, l'un plus petit et l'autre plus grand, entre lesquels se situe ce nombre.
\end{definitionbox}

\begin{examplebox}
	\textbf{Exemples d'encadrements :}
	\begin{itemize}
		\item Donner un encadrement à l'unité de 12,27 : \trous{4cm} < 12,27 < \trous{4cm}
		
		On lit << \trous{5cm} >>
		
		On veut encadrer 12,27 entre deux nombres dont la différence est une unité.
		
		D'autres réponses sont justes : \trous{6cm}
		
		\item Donner un encadrement au dixième de 3,526 : \trous{4cm} < 3,526 < \trous{4cm}
		
		On lit << \trous{5cm} >>
		
		On veut encadrer 3,526 entre deux nombres dont la différence est un dixième.
		
		\item Donner un encadrement au centième de 1,159 : \trous{4cm} < 1,159 < \trous{4cm}
		
		\item Donner un encadrement au millième de 7,1459 : \trous{4cm} < 7,1459 < \trous{4cm}
	\end{itemize}
\end{examplebox}

\section{Lire l'abscisse décimale d'un point}

\textbf{Méthode :} Pour lire l'abscisse d'un point sur une demi-droite graduée, il faut :
\begin{enumerate}
	\item Identifier l'unité et voir en combien de parts elle est divisée
	\item Calculer la valeur de chaque graduation
	\item Compter les graduations depuis l'origine
\end{enumerate}

\begin{examplebox}
	\textbf{Exemple 1 :}
	
	% Schéma avec points à compléter
	\begin{center}
		\begin{tikzpicture}[scale=1.2]
			% Droite graduée de 3 à 4
			\draw[thick, ->] (0,0) -- (10,0);
			
			% Graduations principales
			\foreach \x in {0,10} {
				\draw (\x,-0.2) -- (\x,0.2);
				\node[below] at (\x,-0.4) {\ifnum\x=0 3\else 4\fi};
			}
			
			% Subdivisions (10 intervalles)
			\foreach \x in {1,2,...,9} {
				\draw (\x,-0.1) -- (\x,0.1);
			}
			
			% Points exemple
			\fill[blue] (7,0) circle (0.1);
			\node[above] at (7,0.3) {A};
			
			\fill[red] (2,0) circle (0.1);
			\node[above] at (2,0.3) {B};
			
			\fill[green] (9,0) circle (0.1);
			\node[above] at (9,0.3) {C};
		\end{tikzpicture}
	\end{center}
	
	L'unité a été partagée en \trous{2cm} intervalles : chaque intervalle mesure donc \trous{3cm} de longueur.
	
	L'abscisse du point A est \trous{2cm}
	
	L'abscisse du point B est \trous{2cm}
	
	L'abscisse du point C est \trous{2cm}
	
	On peut noter : A(\trous{2cm}), B(\trous{2cm}), C(\trous{2cm})
	
	1 ÷ 10 = \trous{2cm}
\end{examplebox}

\begin{examplebox}
	\textbf{Exemple 2 :}
	
	% Schéma avec points entre 5,6 et 5,7
	\begin{center}
		\begin{tikzpicture}[scale=1.2]
			% Droite graduée de 5,6 à 5,7
			\draw[thick, ->] (0,0) -- (10,0);
			
			% Graduations principales
			\foreach \x in {0,10} {
				\draw (\x,-0.2) -- (\x,0.2);
				\node[below] at (\x,-0.4) {\ifnum\x=0 5,6\else 5,7\fi};
			}
			
			% Subdivisions (10 intervalles)
			\foreach \x in {1,2,...,9} {
				\draw (\x,-0.1) -- (\x,0.1);
			}
			
			% Points exemple
			\fill[blue] (6,0) circle (0.1);
			\node[above] at (6,0.3) {D};
			
			\fill[red] (1,0) circle (0.1);
			\node[above] at (1,0.3) {E};
			
			\fill[green] (8,0) circle (0.1);
			\node[above] at (8,0.3) {F};
		\end{tikzpicture}
	\end{center}
	
	Un dixième d'unité a été partagé en \trous{2cm} intervalles : chaque intervalle mesure donc \trous{3cm} de longueur.
	
	L'abscisse du point D est \trous{2cm}
	
	L'abscisse du point E est \trous{2cm}
	
	L'abscisse du point F est \trous{2cm}
	
	On peut noter : D(\trous{2cm}), E(\trous{2cm}), F(\trous{2cm})
	
	0,1 ÷ 10 = \trous{2cm}
\end{examplebox}

\section{Lire l'abscisse d'un point par agrandissements successifs}

\textbf{Méthode des agrandissements successifs :}

Cette méthode consiste à << zoomer >> progressivement sur la partie de la droite graduée qui nous intéresse pour lire une abscisse avec plus de précision.

\begin{examplebox}
	% Schémas d'agrandissements successifs à compléter
	\textbf{Étape 1 :} Une centaine a été partagée en 10 : chaque graduation correspond donc à \trous{4cm}
	
	L'abscisse du point A est comprise entre \trous{3cm} et \trous{3cm}
	
	\textbf{Étape 2 :} Une dizaine a été partagée en 10 : chaque graduation correspond donc à \trous{4cm}
	
	L'abscisse du point A est comprise entre \trous{3cm} et \trous{3cm}
	
	\textbf{Étape 3 :} Une unité a été partagée en 10 : chaque graduation correspond donc à \trous{4cm}
	
	L'abscisse du point A est comprise entre \trous{3cm} et \trous{3cm}
	
	\textbf{Étape 4 :} Un dixième a été partagé en 10 : chaque graduation correspond donc à \trous{4cm}
	
	L'abscisse du point A est \trous{3cm}
\end{examplebox}

\vspace{2cm}

\textbf{Exercices d'application :}

\begin{exercisebox}
	1. Écrire sous forme de fraction décimale puis sous forme décimale :
	\begin{itemize}
		\item Trois dixièmes : \trous{3cm} = \trous{2cm}
		\item Vingt-sept centièmes : \trous{3cm} = \trous{2cm}
		\item Cent quarante-cinq millièmes : \trous{3cm} = \trous{2cm}
	\end{itemize}
	
	2. Comparer les nombres suivants (utiliser les symboles <, > ou =) :
	\begin{itemize}
		\item 12,3 \trous{1cm} 12,30
		\item 5,67 \trous{1cm} 5,7
		\item 8,09 \trous{1cm} 8,1
	\end{itemize}
	
	3. Ranger dans l'ordre croissant : 2,1 ; 2,01 ; 2,11 ; 2,101
	
	\trous{8cm}
	
	4. Donner un encadrement au dixième de 7,384 :
	
	\trous{5cm} < 7,384 < \trous{5cm}
\end{exercisebox}