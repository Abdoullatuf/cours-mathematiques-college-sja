% ================================
% PRÉAMBULE COMMUN — Mathématiques (6e, 5e, 4e)
% ================================

% Encodage et langue
\usepackage[T1]{fontenc}
\usepackage[french]{babel}

% Typographie moderne
\usepackage{newtxtext,newtxmath}

% Configuration des pages
\usepackage{geometry}
\geometry{top=2.5cm, bottom=2.5cm, left=2.5cm, right=2.5cm}

% Amélioration du rendu
\usepackage{microtype}

% Personnalisation des titres
\usepackage{titlesec}
\titleformat{\chapter}[hang]{\huge\bfseries}{\thechapter.}{0.6em}{}
\titleformat{\section}[hang]{\Large\bfseries}{\thesection}{0.6em}{}
\titleformat{\subsection}[hang]{\large\bfseries}{\thesubsection}{0.6em}{}

% Ajustement de l'espacement des titres
\titlespacing{\chapter}{0pt}{-10pt}{20pt}
\titlespacing{\section}{0pt}{12pt}{8pt}
\titlespacing{\subsection}{0pt}{4pt}{2pt}

% En-têtes et pieds de page
\usepackage{fancyhdr}

% Packages mathématiques
\usepackage{amsmath, mathtools}
\usepackage{siunitx}

% Graphiques
\usepackage{tikz}
\usetikzlibrary{angles, quotes, calc, arrows.meta, shapes.geometric}
\usepackage{pgfplots}
\pgfplotsset{compat=1.18}

% Icônes modernes
\usepackage{fontawesome5}        % Pour les icônes (Font Awesome 5)

% Liens hypertexte
\usepackage{hyperref}
\hypersetup{
    colorlinks=true,
    linkcolor=blue!60!black,
    urlcolor=blue!60!black,
    citecolor=blue!60!black
}

% Boîtes colorées avec tcolorbox
\usepackage[most]{tcolorbox}

% Palette de couleurs HTML (plus précise et harmonieuse)
\definecolor{mainOrange}{HTML}{F57C00} % Définition
\definecolor{mainGreen}{HTML}{009688}  % Exemple, Démo
\definecolor{mainPurple}{HTML}{8E44AD} % Exercice
\definecolor{mainTeal}{HTML}{00796B}   % Objectifs
\definecolor{mainRed}{HTML}{D32F2F}    % Propriété, Théorème
\definecolor{mainBlue}{HTML}{0077B6}   % Activité
\definecolor{mainYellow}{HTML}{FFA000} % Remarque
\definecolor{mainCyan}{HTML}{00ACC1}   % Quiz
\definecolor{mainGray}{HTML}{546E7A}   % Méthode

% Configuration globale des boîtes
\tcbset{
    rounded corners,
    boxsep=2ex,
    top=1.5ex,
    bottom=1.5ex,
    left=2.5ex,
    right=7ex,
    before skip=0pt,
    after skip=1.5ex,
    width=\textwidth,
    boxrule=1pt
}

% Style commun optimisé pour toutes les boîtes
\tcbset{
    commonstyle/.style={
        boxrule=1pt,
        rounded corners,
        boxsep=2ex, top=1.5ex, bottom=1.5ex, left=2.5ex, right=7ex,
        before skip=0pt, after skip=1.5ex,
        width=\textwidth,
        fonttitle=\bfseries,
        coltitle=white,
        colbacktitle=#1,
        colframe=#1!75!black,
        colback=#1!5!white,
    }
}



% Définition des environnements personnalisés
\newtcolorbox{definitionbox}[1][]{
    commonstyle=mainOrange,
    title={\textbf{Définition \faHighlighter \ \thechapter.\number\numexpr\thedefinition+1\relax}\ifthenelse{\equal{#1}{}}{}{ : #1}},
    before upper={\incdefinition}
}

\newtcolorbox{examplebox}[1][]{
    commonstyle=mainGreen,
    title={\textbf{Exemple \faLightbulb \ \thechapter.\number\numexpr\theexemple+1\relax}\ifthenelse{\equal{#1}{}}{}{ : #1}},
    before upper={\incexemple}
}

\newtcolorbox{exercisebox}{
    commonstyle=mainPurple,
    title={\textbf{Exercices \faPenFancy}}
}

\newtcolorbox{objectifsbox}{
    commonstyle=mainTeal,
    title={\textbf{Objectifs \faBullseye}}
}

\newtcolorbox{proprietebox}[1][]{
    commonstyle=mainRed,
    title={\textbf{Propriété \faStar \ \thechapter.\number\numexpr\thepropriete+1\relax}\ifthenelse{\equal{#1}{}}{}{ : #1}},
    before upper={\incpropriete}
}

\newtcolorbox{theoremebox}[1][]{
    commonstyle=mainRed,
    title={\textbf{Théorème \faCrown \ \thechapter.\number\numexpr\thetheoreme+1\relax}\ifthenelse{\equal{#1}{}}{}{ : #1}},
    before upper={\inctheoreme}
}

\newtcolorbox{activitybox}{
    commonstyle=mainBlue,
    title={\textbf{Activité \faHandsHelping}}
}

\newtcolorbox{remarkbox}[1][]{
    commonstyle=mainYellow,
    title={\textbf{Remarque \faExclamationTriangle \ \thechapter.\number\numexpr\theremarque+1\relax}\ifthenelse{\equal{#1}{}}{}{ : #1}},
    before upper={\incremarque}
}

\newtcolorbox{quizbox}{
    commonstyle=mainCyan,
    title={\textbf{Quiz \faQuestionCircle}}
}

\newtcolorbox{methodebox}[1][]{
    commonstyle=mainGray,
    title={\textbf{Méthode \faRoute \ \thechapter.\number\numexpr\themethode+1\relax}\ifthenelse{\equal{#1}{}}{}{ : #1}},
    before upper={\incmethode}
}

\newtcolorbox{demobox}[1][]{
    commonstyle=mainGreen,
    title={\textbf{Démonstration \faCalculator \ \thechapter.\number\numexpr\thedemo+1\relax}\ifthenelse{\equal{#1}{}}{}{ : #1}},
    before upper={\incdemo}
}

% Compteurs pour les environnements (par chapitre)
\newcounter{definition}
\newcounter{propriete}
\newcounter{methode}
\newcounter{remarque}
\newcounter{exemple}
\newcounter{theoreme}
\newcounter{demo}

% Réinitialisation des compteurs au début de chaque chapitre
\makeatletter
\@addtoreset{definition}{chapter}
\@addtoreset{propriete}{chapter}
\@addtoreset{methode}{chapter}
\@addtoreset{remarque}{chapter}
\@addtoreset{exemple}{chapter}
\@addtoreset{theoreme}{chapter}
\@addtoreset{demo}{chapter}
\makeatother

% Commandes pour incrémenter automatiquement les compteurs
\newcommand{\incdefinition}{\stepcounter{definition}}
\newcommand{\incpropriete}{\stepcounter{propriete}}
\newcommand{\incmethode}{\stepcounter{methode}}
\newcommand{\incremarque}{\stepcounter{remarque}}
\newcommand{\incexemple}{\stepcounter{exemple}}
\newcommand{\inctheoreme}{\stepcounter{theoreme}}
\newcommand{\incdemo}{\stepcounter{demo}}

% Commande personnalisée pour les trous
\newcommand{\trous}[1]{\makebox[#1]{\rule{0pt}{1.2ex}\dotfill}}

% Variable pour stocker le titre de la séquence
\newcommand{\seqtitle}{}
\newcommand{\setseqtitle}[1]{\renewcommand{\seqtitle}{#1}}

% Mise en page de l'en-tête et du pied
\pagestyle{fancy}
\setlength{\headheight}{16pt} % FIX: évite le warning fancyhdr
\fancyhf{}
\lhead{Mathématiques \niveau{} -- 2025--2026}
\rhead{Seq.~\thechapter~--~\seqtitle}
\cfoot{\thepage}

% Listes compactes
\usepackage{enumitem}
\setlist[itemize]{left=1.2em}
\setlist[enumerate]{left=1.5em}

% Configuration des labels personnalisés pour enumitem
\SetEnumitemKey{a}{label=\alph*)}
\SetEnumitemKey{1}{label=\arabic*)}

% Définition des styles de listes personnalisés
\newlist{exerciselist}{enumerate}{1}
\setlist[exerciselist]{label=\alph*)}

\newlist{quizlist}{enumerate}{1}
\setlist[quizlist]{label=\arabic*)}

% Packages pour tableaux et présentations
\usepackage{longtable}
\usepackage{array}
\usepackage{booktabs}

% Logique conditionnelle pour les titres optionnels
\usepackage{ifthen}
