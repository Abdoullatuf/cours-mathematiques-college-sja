% ================================
% PRÉAMBULE COMMUN — Mathématiques (6e, 5e, 4e)
% ================================

% Encodage et langue
\usepackage[T1]{fontenc}
\usepackage[french]{babel}

% Typographie moderne
\usepackage{newtxtext,newtxmath}

% Configuration des pages
\usepackage{geometry}
\geometry{top=2.5cm, bottom=2.5cm, left=2.5cm, right=2.5cm}

% Amélioration du rendu
\usepackage{microtype}

% Personnalisation des titres
\usepackage{titlesec}
\titleformat{\chapter}[hang]{\huge\bfseries}{\thechapter.}{0.6em}{}
\titleformat{\section}[hang]{\Large\bfseries}{\thesection}{0.6em}{}
\titleformat{\subsection}[hang]{\large\bfseries}{\thesubsection}{0.6em}{}

% Ajustement de l'espacement des titres
\titlespacing{\chapter}{0pt}{-10pt}{20pt}
\titlespacing{\section}{0pt}{12pt}{8pt}
\titlespacing{\subsection}{0pt}{4pt}{2pt}

% En-têtes et pieds de page
\usepackage{fancyhdr}

% Packages mathématiques
\usepackage{amsmath, mathtools}
\usepackage{siunitx}

% Graphiques
\usepackage{tikz}
\usetikzlibrary{angles, quotes, calc, arrows.meta, shapes.geometric}
\usepackage{pgfplots}
\pgfplotsset{compat=1.18}

% Liens hypertexte
\usepackage{hyperref}
\hypersetup{
    colorlinks=true,
    linkcolor=blue!60!black,
    urlcolor=blue!60!black,
    citecolor=blue!60!black
}

% Boîtes colorées avec tcolorbox
\usepackage[most]{tcolorbox}

% Configuration globale des boîtes
\tcbset{
    rounded corners,
    boxsep=2ex,
    top=1.5ex,
    bottom=1.5ex,
    left=2.5ex,
    right=7ex,
    before skip=0pt,
    after skip=1.5ex,
    width=\textwidth,
    boxrule=1pt
}



% Définition des environnements personnalisés
\newtcolorbox{definitionbox}[1][]{
    colback=orange!5!white,
    colframe=orange!70!black,
    title={\textbf{Définition \thechapter.\number\numexpr\thedefinition+1\relax}\ifthenelse{\equal{#1}{}}{}{ : #1}},
    fonttitle=\bfseries,
    coltitle=black,
    before upper={\incdefinition}
}

\newtcolorbox{examplebox}[1][]{
    colback=green!5!white,
    colframe=green!60!black,
    title={\textbf{Exemple \thechapter.\number\numexpr\theexemple+1\relax}\ifthenelse{\equal{#1}{}}{}{ : #1}},
    fonttitle=\bfseries,
    coltitle=black,
    before upper={\incexemple}
}

\newtcolorbox{exercisebox}{
    colback=purple!5!white,
    colframe=purple!70!black,
    title={\textbf{Exercices}},
    fonttitle=\bfseries,
    coltitle=black
}

\newtcolorbox{objectifsbox}{
    colback=teal!5!white,
    colframe=teal!70!black,
    title={\textbf{Objectifs}},
    fonttitle=\bfseries,
    coltitle=black
}

\newtcolorbox{proprietebox}[1][]{
    colback=red!5!white,
    colframe=red!70!black,
    title={\textbf{Propriété \thechapter.\number\numexpr\thepropriete+1\relax}\ifthenelse{\equal{#1}{}}{}{ : #1}},
    fonttitle=\bfseries,
    coltitle=black,
    before upper={\incpropriete}
}

\newtcolorbox{theoremebox}[1][]{
    colback=red!5!white,
    colframe=red!70!black,
    title={\textbf{Théorème \thechapter.\number\numexpr\thetheoreme+1\relax}\ifthenelse{\equal{#1}{}}{}{ : #1}},
    fonttitle=\bfseries,
    coltitle=black,
    before upper={\inctheoreme}
}

\newtcolorbox{activitybox}{
    colback=blue!5!white,
    colframe=blue!70!black,
    title={\textbf{Activité}},
    fonttitle=\bfseries,
    coltitle=black
}

\newtcolorbox{remarkbox}[1][]{
    colback=yellow!5!white,
    colframe=yellow!70!black,
    title={\textbf{Remarque \thechapter.\number\numexpr\theremarque+1\relax}\ifthenelse{\equal{#1}{}}{}{ : #1}},
    fonttitle=\bfseries,
    coltitle=black,
    before upper={\incremarque}
}

\newtcolorbox{quizbox}{
    colback=cyan!5!white,
    colframe=cyan!70!black,
    title={\textbf{Quiz}},
    fonttitle=\bfseries,
    coltitle=black
}

\newtcolorbox{methodebox}[1][]{
    colback=purple!10!white,
    colframe=purple!50!black,
    title={\textbf{Méthode \thechapter.\number\numexpr\themethode+1\relax}\ifthenelse{\equal{#1}{}}{}{ : #1}},
    fonttitle=\bfseries,
    coltitle=black,
    before upper={\incmethode}
}

% Compteurs pour les environnements (par chapitre)
\newcounter{definition}
\newcounter{propriete}
\newcounter{methode}
\newcounter{remarque}
\newcounter{exemple}
\newcounter{theoreme}

% Réinitialisation des compteurs au début de chaque chapitre
\makeatletter
\@addtoreset{definition}{chapter}
\@addtoreset{propriete}{chapter}
\@addtoreset{methode}{chapter}
\@addtoreset{remarque}{chapter}
\@addtoreset{exemple}{chapter}
\@addtoreset{theoreme}{chapter}
\makeatother

% Commandes pour incrémenter automatiquement les compteurs
\newcommand{\incdefinition}{\stepcounter{definition}}
\newcommand{\incpropriete}{\stepcounter{propriete}}
\newcommand{\incmethode}{\stepcounter{methode}}
\newcommand{\incremarque}{\stepcounter{remarque}}
\newcommand{\incexemple}{\stepcounter{exemple}}
\newcommand{\inctheoreme}{\stepcounter{theoreme}}

% Commande personnalisée pour les trous
\newcommand{\trous}[1]{\makebox[#1]{\rule{0pt}{1.2ex}\dotfill}}

% Variable pour stocker le titre de la séquence
\newcommand{\seqtitle}{}
\newcommand{\setseqtitle}[1]{\renewcommand{\seqtitle}{#1}}

% Mise en page de l'en-tête et du pied
\pagestyle{fancy}
\setlength{\headheight}{16pt} % FIX: évite le warning fancyhdr
\fancyhf{}
\lhead{Mathématiques \niveau{} -- 2025--2026}
\rhead{Seq.~\thechapter~--~\seqtitle}
\cfoot{\thepage}

% Listes compactes
\usepackage{enumitem}
\setlist[itemize]{left=1.2em}
\setlist[enumerate]{left=1.5em}

% Configuration des labels personnalisés pour enumitem
\SetEnumitemKey{a}{label=\alph*)}
\SetEnumitemKey{1}{label=\arabic*)}

% Définition des styles de listes personnalisés
\newlist{exerciselist}{enumerate}{1}
\setlist[exerciselist]{label=\alph*)}

\newlist{quizlist}{enumerate}{1}
\setlist[quizlist]{label=\arabic*)}

% Packages pour tableaux et présentations
\usepackage{longtable}
\usepackage{array}
\usepackage{booktabs}
